% Options for packages loaded elsewhere
\PassOptionsToPackage{unicode}{hyperref}
\PassOptionsToPackage{hyphens}{url}
\PassOptionsToPackage{dvipsnames,svgnames,x11names}{xcolor}
%
\documentclass[
]{article}
\usepackage{amsmath,amssymb}
\usepackage{lmodern}
\usepackage{setspace}
\usepackage{iftex}
\ifPDFTeX
  \usepackage[T1]{fontenc}
  \usepackage[utf8]{inputenc}
  \usepackage{textcomp} % provide euro and other symbols
\else % if luatex or xetex
  \usepackage{unicode-math}
  \defaultfontfeatures{Scale=MatchLowercase}
  \defaultfontfeatures[\rmfamily]{Ligatures=TeX,Scale=1}
\fi
% Use upquote if available, for straight quotes in verbatim environments
\IfFileExists{upquote.sty}{\usepackage{upquote}}{}
\IfFileExists{microtype.sty}{% use microtype if available
  \usepackage[]{microtype}
  \UseMicrotypeSet[protrusion]{basicmath} % disable protrusion for tt fonts
}{}
\makeatletter
\@ifundefined{KOMAClassName}{% if non-KOMA class
  \IfFileExists{parskip.sty}{%
    \usepackage{parskip}
  }{% else
    \setlength{\parindent}{0pt}
    \setlength{\parskip}{6pt plus 2pt minus 1pt}}
}{% if KOMA class
  \KOMAoptions{parskip=half}}
\makeatother
\usepackage{xcolor}
\IfFileExists{xurl.sty}{\usepackage{xurl}}{} % add URL line breaks if available
\IfFileExists{bookmark.sty}{\usepackage{bookmark}}{\usepackage{hyperref}}
\hypersetup{
  pdftitle={A general model for the evolution of nuptial gift-giving},
  colorlinks=true,
  linkcolor={blue},
  filecolor={Maroon},
  citecolor={Blue},
  urlcolor={Blue},
  pdfcreator={LaTeX via pandoc}}
\urlstyle{same} % disable monospaced font for URLs
\usepackage[margin=1in]{geometry}
\usepackage{color}
\usepackage{fancyvrb}
\newcommand{\VerbBar}{|}
\newcommand{\VERB}{\Verb[commandchars=\\\{\}]}
\DefineVerbatimEnvironment{Highlighting}{Verbatim}{commandchars=\\\{\}}
% Add ',fontsize=\small' for more characters per line
\usepackage{framed}
\definecolor{shadecolor}{RGB}{248,248,248}
\newenvironment{Shaded}{\begin{snugshade}}{\end{snugshade}}
\newcommand{\AlertTok}[1]{\textcolor[rgb]{0.94,0.16,0.16}{#1}}
\newcommand{\AnnotationTok}[1]{\textcolor[rgb]{0.56,0.35,0.01}{\textbf{\textit{#1}}}}
\newcommand{\AttributeTok}[1]{\textcolor[rgb]{0.77,0.63,0.00}{#1}}
\newcommand{\BaseNTok}[1]{\textcolor[rgb]{0.00,0.00,0.81}{#1}}
\newcommand{\BuiltInTok}[1]{#1}
\newcommand{\CharTok}[1]{\textcolor[rgb]{0.31,0.60,0.02}{#1}}
\newcommand{\CommentTok}[1]{\textcolor[rgb]{0.56,0.35,0.01}{\textit{#1}}}
\newcommand{\CommentVarTok}[1]{\textcolor[rgb]{0.56,0.35,0.01}{\textbf{\textit{#1}}}}
\newcommand{\ConstantTok}[1]{\textcolor[rgb]{0.00,0.00,0.00}{#1}}
\newcommand{\ControlFlowTok}[1]{\textcolor[rgb]{0.13,0.29,0.53}{\textbf{#1}}}
\newcommand{\DataTypeTok}[1]{\textcolor[rgb]{0.13,0.29,0.53}{#1}}
\newcommand{\DecValTok}[1]{\textcolor[rgb]{0.00,0.00,0.81}{#1}}
\newcommand{\DocumentationTok}[1]{\textcolor[rgb]{0.56,0.35,0.01}{\textbf{\textit{#1}}}}
\newcommand{\ErrorTok}[1]{\textcolor[rgb]{0.64,0.00,0.00}{\textbf{#1}}}
\newcommand{\ExtensionTok}[1]{#1}
\newcommand{\FloatTok}[1]{\textcolor[rgb]{0.00,0.00,0.81}{#1}}
\newcommand{\FunctionTok}[1]{\textcolor[rgb]{0.00,0.00,0.00}{#1}}
\newcommand{\ImportTok}[1]{#1}
\newcommand{\InformationTok}[1]{\textcolor[rgb]{0.56,0.35,0.01}{\textbf{\textit{#1}}}}
\newcommand{\KeywordTok}[1]{\textcolor[rgb]{0.13,0.29,0.53}{\textbf{#1}}}
\newcommand{\NormalTok}[1]{#1}
\newcommand{\OperatorTok}[1]{\textcolor[rgb]{0.81,0.36,0.00}{\textbf{#1}}}
\newcommand{\OtherTok}[1]{\textcolor[rgb]{0.56,0.35,0.01}{#1}}
\newcommand{\PreprocessorTok}[1]{\textcolor[rgb]{0.56,0.35,0.01}{\textit{#1}}}
\newcommand{\RegionMarkerTok}[1]{#1}
\newcommand{\SpecialCharTok}[1]{\textcolor[rgb]{0.00,0.00,0.00}{#1}}
\newcommand{\SpecialStringTok}[1]{\textcolor[rgb]{0.31,0.60,0.02}{#1}}
\newcommand{\StringTok}[1]{\textcolor[rgb]{0.31,0.60,0.02}{#1}}
\newcommand{\VariableTok}[1]{\textcolor[rgb]{0.00,0.00,0.00}{#1}}
\newcommand{\VerbatimStringTok}[1]{\textcolor[rgb]{0.31,0.60,0.02}{#1}}
\newcommand{\WarningTok}[1]{\textcolor[rgb]{0.56,0.35,0.01}{\textbf{\textit{#1}}}}
\usepackage{longtable,booktabs,array}
\usepackage{calc} % for calculating minipage widths
% Correct order of tables after \paragraph or \subparagraph
\usepackage{etoolbox}
\makeatletter
\patchcmd\longtable{\par}{\if@noskipsec\mbox{}\fi\par}{}{}
\makeatother
% Allow footnotes in longtable head/foot
\IfFileExists{footnotehyper.sty}{\usepackage{footnotehyper}}{\usepackage{footnote}}
\makesavenoteenv{longtable}
\usepackage{graphicx}
\makeatletter
\def\maxwidth{\ifdim\Gin@nat@width>\linewidth\linewidth\else\Gin@nat@width\fi}
\def\maxheight{\ifdim\Gin@nat@height>\textheight\textheight\else\Gin@nat@height\fi}
\makeatother
% Scale images if necessary, so that they will not overflow the page
% margins by default, and it is still possible to overwrite the defaults
% using explicit options in \includegraphics[width, height, ...]{}
\setkeys{Gin}{width=\maxwidth,height=\maxheight,keepaspectratio}
% Set default figure placement to htbp
\makeatletter
\def\fps@figure{htbp}
\makeatother
\setlength{\emergencystretch}{3em} % prevent overfull lines
\providecommand{\tightlist}{%
  \setlength{\itemsep}{0pt}\setlength{\parskip}{0pt}}
\setcounter{secnumdepth}{-\maxdimen} % remove section numbering
\newlength{\cslhangindent}
\setlength{\cslhangindent}{1.5em}
\newlength{\csllabelwidth}
\setlength{\csllabelwidth}{3em}
\newlength{\cslentryspacingunit} % times entry-spacing
\setlength{\cslentryspacingunit}{\parskip}
\newenvironment{CSLReferences}[2] % #1 hanging-ident, #2 entry spacing
 {% don't indent paragraphs
  \setlength{\parindent}{0pt}
  % turn on hanging indent if param 1 is 1
  \ifodd #1
  \let\oldpar\par
  \def\par{\hangindent=\cslhangindent\oldpar}
  \fi
  % set entry spacing
  \setlength{\parskip}{#2\cslentryspacingunit}
 }%
 {}
\usepackage{calc}
\newcommand{\CSLBlock}[1]{#1\hfill\break}
\newcommand{\CSLLeftMargin}[1]{\parbox[t]{\csllabelwidth}{#1}}
\newcommand{\CSLRightInline}[1]{\parbox[t]{\linewidth - \csllabelwidth}{#1}\break}
\newcommand{\CSLIndent}[1]{\hspace{\cslhangindent}#1}
\usepackage{amsmath}
\usepackage{natbib}
\usepackage{lineno}
\usepackage{caption}
\usepackage[utf8]{inputenc}
\bibliographystyle{amnatnat}
\ifLuaTeX
  \usepackage{selnolig}  % disable illegal ligatures
\fi

\title{A general model for the evolution of nuptial gift-giving}
\author{Anders P. Charmouh\(^{1,4, 5}\), Trine Bilde\(^{2}\), Greta
Bocedi\(^{1}\), A. Bradley Duthie\(^{3}\)}
\date{{[}1{]} School of Biological Sciences, University of Aberdeen, UK.
{[}2{]} Department of Biology, Aarhus University, Denmark. {[}3{]}
Biological and Environmental Sciences, University of Stirling, UK.
{[}4{]} Bioinformatics Research Centre Aarhus University, University
City 81, building 1872, 3rd floor, DK-8000 Aarhus C, Denmark {[}5{]}
Email: \href{mailto:apc@birc.au.dk}{\nolinkurl{apc@birc.au.dk}}}

\begin{document}
\maketitle

\setstretch{1}
\textbf{Key words:} Nuptial gifts, male search, mate interactions,
modelling, \emph{Pisaura}

\begin{center}\rule{0.5\linewidth}{0.5pt}\end{center}

Nuptial gift-giving occurs in several taxonomic groups including
insects, snails, birds, squid, arachnids and humans. Although this trait
has evolved many times independently, no general framework has been
developed to predict the conditions necessary for nuptial gift-giving to
evolve. We use a time-in time-out model to derive analytical results
describing the requirements necessary for selection to favour nuptial
gift-giving. Specifically, selection will favour nuptial gift-giving if
the fitness increase caused by gift-giving exceeds the product of
expected gift search time and encounter rate of the opposite sex.
Selection will favour choosiness in the opposite sex if the value of a
nuptial gift exceeds the inverse of the time taken to produce offspring
multiplied by the rate at which mates with nuptial gifts are
encountered. Importantly, selection can differ between the sexes,
potentially causing sexual conflict. We test these results using an
individual-based model applied to a system of nuptial gift-giving
spiders, \emph{Pisaura mirabilis}, by estimating parameter values using
experimental data from several studies. Our results provide a general
framework for understanding when the evolution of nuptial gift-giving
can occur and provide novel insight into the evolution of worthless
nuptial gifts, occurring in multiple taxonomic groups with implications
for understanding parental investment.

\begin{center}\rule{0.5\linewidth}{0.5pt}\end{center}

\hypertarget{introduction}{%
\section{Introduction}\label{introduction}}

Nuptial gift-giving occurs when the choosy sex (usually the female)
receives gifts from the opposite sex (usually the male) during
courtship. It is a widespread phenomenon, occurring within several
diverse taxonomic groups such as insects, snails, birds, squid,
arachnids and humans (\protect\hyperlink{ref-Lewis2012}{Lewis \& South,
2012}; \protect\hyperlink{ref-Albo2014}{Albo \emph{et al.}, 2014};
\protect\hyperlink{ref-Lewis2014}{Lewis \emph{et al.}, 2014}). Despite
the ubiquity of this behaviour, little effort has been made to
conceptualise the evolution of nuptial gift-giving within a general
modelling framework (\protect\hyperlink{ref-Lewis2014}{Lewis \emph{et
al.}, 2014}; \protect\hyperlink{ref-Iwasa2022}{Iwasa \& Yamaguchi,
2022}). Recent models describing the evolution of nuptial gift-giving
have focused on co-evolution between male nuptial gift-giving and female
propensity to remate, and evolutionarily stable nuptial gift sizes
(\protect\hyperlink{ref-Kamimura2021}{Kamimura \emph{et al.}, 2021};
\protect\hyperlink{ref-Iwasa2022}{Iwasa \& Yamaguchi, 2022}), but a
general framework describing the conditions necessary for nuptial
gift-giving to be initially favoured by selection is needed to
understand when gift-giving should evolve.

Nuptial gift-giving may allow males to increase fitness by acquiring
additional mates, indirect benefits (by increasing the fitness of
offspring), prolonged copulations, and success in sperm competition
(\protect\hyperlink{ref-Albo2013}{Albo \emph{et al.}, 2013};
\protect\hyperlink{ref-Ghislandi2014}{Ghislandi \emph{et al.}, 2014};
\protect\hyperlink{ref-Lewis2014}{Lewis \emph{et al.}, 2014}). However,
this potential fitness increase comes at the expense of producing a
nuptial gift, which may be costly in terms of time and resources.
Females may increase their fitness by receiving nutritionally valuable
nuptial gifts, but expressing a preference for males with gifts might
result in a mating opportunity cost if available males without gifts are
rejected. With respect to nuptial gift-giving, the evolutionary
interests of both sexes may not always fully overlap. This can cause
sexual conflict, which is a difference in the evolutionary interest
between sexes that occurs when interaction between sexes results in a
situation where neither sex can achieve an optimal outcome
(\protect\hyperlink{ref-Parker2006}{Parker, 2006}). That is, under some
conditions, it might for example be optimal for males but not females to
mate if females do not benefit from mating with males without nuptial
gifts.

Much work has sought to explain how gift-giving tactics are maintained,
with explanations including condition-dependent strategies, gift-giving
as a way to decrease female aggression during copulation, or gifts as
sensory traps (\protect\hyperlink{ref-Lubin2007}{Lubin \& Bilde, 2007};
\protect\hyperlink{ref-Toft2016}{Toft \& Albo, 2016};
\protect\hyperlink{ref-Ghislandi2018}{Ghislandi \emph{et al.}, 2018};
\protect\hyperlink{ref-Albo2019}{Albo \emph{et al.}, 2019}). An example
of such a system is the nuptial gift-giving nursery-web spider
\emph{Pisaura mirabilis} where males may court females with or without
nuptial gifts (\protect\hyperlink{ref-Bristowe1926}{Bristowe \& Locket,
1926}; \protect\hyperlink{ref-Tuni2013a}{Tuni \emph{et al.}, 2013}).
Here, males may provide females with costly nuptial gifts in the form of
captured arthropod prey, and females may exhibit preference for males
with a nuptial gift by rejecting males without a nuptial gift
(\protect\hyperlink{ref-Albo2013}{Albo \emph{et al.}, 2013}).

We develop a general framework for investigating the evolution of
nuptial gift-giving and choosiness using a time-in, time-out modelling
approach and an individual-based model
(\protect\hyperlink{ref-Clutton-Brock1992}{Clutton-Brock \& Parker,
1992}). Specifically, we derive conditions under which selection will
favour male search for nuptial gifts and female rejection of gift-less
males. We show that selection for searching and choosiness depends on
whether a threshold fitness value of the nuptial gift is exceeded. Our
model demonstrates the importance of nuptial gift cost, sex ratio, and
mate encounter rate in determining the threshold above which selection
will favour the evolution of nuptial gift-giving. Importantly, we show
that the threshold value differs for males and females. We test
predictions of our analytical model by formulating an individual-based
model, which further supports the main theoretical results of our
analytical model. Further, we apply our model to an example system with
nuptial gifts, the nursery web spider \emph{Pisaura mirabilis}, where we
use experimental data to estimate a key model parameter. Our results
provide a general framework for understanding why nuptial gift-giving
evolves in some systems and not in others, how the evolution of nuptial
gift-giving can give rise to sexual conflict, and it provides insight
into the evolution of worthless and deceitful nuptial gifts, which occur
in several different taxonomic (\protect\hyperlink{ref-LeBas2005}{LeBas
\& Hockham, 2005}; \protect\hyperlink{ref-Ghislandi2014}{Ghislandi
\emph{et al.}, 2014}).

\hypertarget{results}{%
\section{Results}\label{results}}

\hypertarget{analytical-model}{%
\subsection{Analytical model}\label{analytical-model}}

We use a time-in and time-out model
(\protect\hyperlink{ref-Clutton-Brock1992}{Clutton-Brock \& Parker,
1992}; \protect\hyperlink{ref-Kokko2001}{Kokko \& Monaghan, 2001};
\protect\hyperlink{ref-Kokko2006}{Kokko \& Ots, 2006}) in which choosy
(female) and non-choosy (male) individuals spend some period of time
within the mating pool searching for a mate (time-in) followed by a
period outside the mating pool (time-out). During time-out, females
spend some duration of time (\(T_{\mathrm{f}}\)) gestating or rearing
(hereafter `processing') offspring. We define the number of offspring
produced by a female per reproductive cycle as \(\lambda\). Since
females enter time-out after mating, this assumption is equivalent to
assuming a system with sequential polyandry. For simplicity, we assume
male time to replenish sperm is negligible, but males can spend some
duration of time (\(T_{\mathrm{m}}\)) out of the mating pool searching
for nuptial gifts.

\begin{figure}
\includegraphics[width=1\linewidth]{inst/img/conceptual_figure} \caption{Conceptual figure inspired by Kokko and Ots (2006) illustrating how the modelling framework maps onto an example of a system wherein nuptial gifts are used, here \textit{Pisaura mirabilis}. Males have a probability of obtaining a nuptial gift while in time-out, which will affect their probability of mating while in time-in. They return to the mating pool (time-in) at a rate determined by the time spent searching for a nuptial gifts ($T_{\mathrm{m}}$) and leave the mating pool (i.e. enter time-out) following the female encounter rate, which is dependent on the ratio of males to females ($\beta$) and the encounter rate ($R$). The choosy sex (females) enter the mating pool at a rate depending on the time spent processing offspring ($T_{\mathrm{f}}$) and leave the mating pool (i.e. enter time-out) at a rate that is dependent on $\beta$ and $R$. Males and females undergo sex-specific mortality $\mu$ during time-in and time-out. Image left to right: (1) male \textit{P. mirabilis}. (2) male \textit{P. mirabilis} presenting nuptial gift (white) to female. (3) Female \textit{P. mirabilis} protecting offspring. Photos: Alamy.}\label{fig:unnamed-chunk-2}
\end{figure}

\hypertarget{criteria-for-male-search-and-female-choosiness}{%
\subsection{Criteria for male search and female
choosiness}\label{criteria-for-male-search-and-female-choosiness}}

The probability \(G\) that a male succeeds in securing a nuptial gift is
defined by,

\begin{equation}
G = 1 - e^{-\frac{1}{\alpha}T_{\mathrm{m}}}.
\end{equation}

In Eq. 1, \(\alpha\) defines the expected search time before
encountering a nuptial gift. Thus, the probability of finding a nuptial
gift is higher the more time \(T_{\mathrm{m}}\) is spent searching.
During time-in, individuals encounter conspecifics at a rate of \(R\). A
focal individual will therefore encounter conspecifics of the opposite
sex at a rate of \(R/2\) if the ratio of males to females in the mating
pool (\(\beta\)) is equal. More generally, males will be encountered at
a rate of \(R\beta/(\beta+1)\), and females will be encountered at a
rate of \(R/(\beta+1)\). An example of how the structure of the time-in
time-out model applies to a system with nuptial gift-giving is given in
Figure 1. We assume that mating with a nuptial gift increases the
fitness of each offspring by an increment of \(\gamma\). We proceed to
find the thresholds \(\gamma_{\mathrm{m}}\) and \(\gamma_{\mathrm{f}}\)
above which males and females are favoured by selection to search for
mates with nuptial gifts and exhibit choosiness for nuptial gifts,
respectively. We show (see Methods) that the initial threshold value of
\(\gamma\) (\(\gamma_{\mathrm{m}}\)) necessary for males to increase
their fitness by investing time searching for a nuptial gift (time that
could otherwise be spent searching for a mate) is,

\begin{equation}
\gamma_{\mathrm{m}} > \alpha \frac{R}{\beta + 1}.
\end{equation}

Inequality 2 means that if nuptial gifts are not abundant and thus
require a long time to find (i.e., high \(\alpha\)), or if males
encounter many females per unit time (i.e., high \(R / (1+\beta)\)),
then the nuptial gift must result in a high fitness increment for
selection to favour gift-searching. In general, when ineq. 2 is
satisfied, we predict selection to favour the evolution of nuptial
gift-giving.

We can similarly predict the conditions for which there is selection for
female choosiness. If \(\gamma\) is sufficiently high, then females
increase their fitness by rejecting males without gifts and mating only
with males that provide nuptial gifts. To illustrate, we assume that all
males in a population search for a duration of \(T_{\mathrm{m}}\), in
which case the threshold fitness increment for females
(\(\gamma_{\mathrm{f}}\)) is,

\begin{equation}
\gamma_{\mathrm{f}} > \frac{1}{T_{\mathrm{f}} R \left(\frac{\beta}{\beta + 1}\right) \left(1 - e^{-\frac{1}{\alpha}T_{\mathrm{m}}} \right)}.
\end{equation}

Inequality 3 shows that as offspring processing time
(\(T_{\mathrm{f}}\)), mate encounter rate (\(R\beta / (\beta + 1)\)), or
the probability of a male finding a nuptial gift
(\(1 - \exp(-T_{\mathrm{m}}/\alpha)\) ) decreases, the threshold value
of fitness above which selection will favour choosiness
(\(\gamma_{\mathrm{f}}\)) increases. This can be understood intuitively
by realising that rejecting a prospective male represents an opportunity
cost for the female. This opportunity cost becomes small if many males
with gifts are encountered, hence the appearance of the rate at which
males with gifts are encountered in the denominator. Figure 2 shows how
\(\gamma_{\mathrm{m}}\) and \(\gamma_{\mathrm{f}}\) change with
increasing \(\alpha\). For \(\gamma_{\mathrm{f}}\), we assume that males
search for the expected time required to obtain a nuptial gift
(\(T_{\mathrm{m}} = \alpha\)). Note that \(\beta\) does not have a
closed form solution given \(R\), \(T_{\mathrm{f}}\), and
\(T_{\mathrm{m}}\), so \(\beta\) was calculated using recursion (see
Methods).

\begin{figure}
\centering
\includegraphics{ms_biorxiv_files/figure-latex/unnamed-chunk-3-1.pdf}
\caption{Fitness thresholds above which males increase their fitness by
searching for nuptial gifts (blue lines; Eq. 2) and females increase
their fitness by rejecting males that do not offer gifts (red lines; Eq.
3). Parameter space includes areas in which males do not search for
nuptial gifts and females are not choosy (A), males search but females
are not choosy (B), females would be choosy but males do not search (C),
and males search and females are choosy (D). Arrows in panel a indicate
the effect of increasing interaction rate (\(R\)), female time-out
(\(T_{\mathrm{f}}\)), and male search time (\(T_{\mathrm{m}}\)). As an
example, trajectories for \(T_{\mathrm{f}} = 2\), and
\(T_{\mathrm{m}} = \alpha\) are shown for values of \(R=2\) (panel a)
and \(R = 1\) (panel b). Females are assumed to be the choosy sex, which
is maintained as long as \(\alpha < T_{\mathrm{f}}\).}
\end{figure}

The analytical framework predicts 4 zones, which are delineated by
inequalities 2 and 3 and describe the initial thresholds for favouring
search of nuptial gifts in males and choosiness for nuptial gifts in
females (Figure 2a). Consequently, the modelling framework gives a
description of the conditions under which nuptial gift-giving is
expected to occur (Figure 2a, Zone D) and the conditions under which
only selection for male searching (Figure 2a, Zone B) or female
choosiness (Figure 2a, Zone C) are predicted. These results therefore
highlight the potential for sexual conflict over nuptial gift-giving.

\hypertarget{evolution-of-male-search-and-female-choosiness}{%
\subsection{Evolution of male search and female
choosiness}\label{evolution-of-male-search-and-female-choosiness}}

We used an individual-based model (IBM) to simulate the evolution of
nuptial gift-giving and female choosiness from an ancestral condition in
which neither exists. The IBM was written to satisfy the assumptions of
our analytical time-in and time-out model as much as practical (see
Supporting Information S1). Using the IBM, we modelled a
spatially-implicit, finite population of females and males. At each time
step, some individuals enter or remain within the mating pool (time-in),
where they potentially interact and mate. After mating, males and
females may leave the mating pool to search for nuptial gifts and to
produce offspring, respectively (time-out). Mortality occurs with a
fixed probability in each time step, then a ceiling regulation is
applied to limit population growth (see Methods).

The rates at which females are encountered by males
\(R_{\mathrm{f},\mathrm{m}}\) and males with nuptial gifts
\(R_{\mathrm{m_{G}}, \mathrm{f}}\) are encountered by females are both
calculated directly from the IBM, thereby modelling how these rates
might be estimated from empirical data, so the male threshold for
increasing fitness by searching is,

\begin{equation}
\gamma_{\mathrm{m, IBM}} > \alpha R_{\mathrm{f},\mathrm{m}}.
\end{equation}

Similarly, female threshold for increasing fitness by choosiness is,

\begin{equation}
\gamma_{\mathrm{f, IBM}} > \frac{1}{T_{\mathrm{f}}R_{\mathrm{m_{G}}, \mathrm{f}}}.
\end{equation}

Consequently, the IBM and the analytical model differ slightly (e.g.,
time is discrete in the IBM but continuous the analytical model, and in
the IBM, a fitness increment is applied to the focal female in the form
of birthrate increase rather than offspring fitness; see Supporting
Information S1 for details). But the predicted thresholds are
theoretically equivalent and yield predictions that are qualitatively
the same (Figure 3).

\begin{figure}
\centering
\includegraphics{ms_biorxiv_files/figure-latex/unnamed-chunk-4-1.pdf}
\caption{The coevolution of male search and female choosiness as a
function of nuptial gift search time (\(\alpha\)). Points show where the
lower 95\% confidence interval of female choosiness (red) and male
search (blue) exceeds zero, indicating evolution of choosiness or
nuptial gift search. Each point includes data from 3200 replicate
simulations with identical starting conditions. Red and blue lines show
thresholds above which the mathematical model predicts that females
should be choosy and males should search, respectively (in agreement
with Figure 2). Minor deviations between the analytical model and the
simulation results are expected due to the finite population size, the
substantial stochasticity inherent to the simulation model, and the
potential for coevolution (for details, see Supporting Information S1).
The number of individuals in the population remained at or near carrying
capacity of \(K = 1000\). In each time step, up to 3000 total pair-wise
interactions occurred. Expected female processing time was set to
\(T_{\mathrm{f}}=2\) time steps, and \(\gamma\) and \(\alpha\) values in
the range {[}0.0, 2.0{]} and {[}0.1, 2.0{]}, respectively, were used.}
\end{figure}

Male thresholds \(\gamma_{\mathrm{m, IBM}}\) given by ineq. 4 accurately
predict the evolution of searching in the IBM across \(\alpha\) values,
and the female threshold \(\gamma_{\mathrm{f, IBM}}\) (ineq. 5) predicts
the evolution of female choice (Figure 3). In other words, IBM
simulations demonstrate that nuptial gift search in males, and
choosiness in females, will evolve from an ancestral state of no
searching and no choosiness in similar parameter space (Figure 3) as
predicted by the analytical model (Figure 2b). We further ran the IBM
with realistic values of \(\gamma\), estimated using data from the
\emph{P. mirabilis} system wherein choosiness among females, and nuptial
gift search among males, occur (Figure 4). We found that our IBM
predicts both the evolution of choosiness and nuptial gift searching
observed in the \emph{P. mirabilis} system.

\begin{figure}
\centering
\includegraphics{ms_biorxiv_files/figure-latex/unnamed-chunk-5-1.pdf}
\caption{The joint evolution of male search and female choosiness using
a nuptial gift fitness increment (\(\gamma\)) that was estimated from
experimental data for the species \emph{Pisaura mirabilis} (mean
\(\gamma = 3.29 \pm\) SE via propagation of error for estimates of
nuptial feeding a non-nuptial feeding groups: {[}1.47; 8.32{]}). Points
show where the 95\% confidence interval exceeds 0 for female choosiness
(red) and male search (blue). Each point includes data from 100
replicate simulations with identical starting conditions. The red line
shows the threshold above which females should be choosy and the blue
line shows the threshold above which males should search. As predicted
by the analytical model, both male search and female choosiness evolved
for a range of \(\gamma\) values around the empirical estimates. This
occurs because these values result in a \(\gamma\) above the fitness
threshold necessary for selection to favour male search of nuptial gifts
(blue line) and female choosiness (red line). The number of individuals
in the population remained at or near carrying capacity of \(K = 1000\).
In each time step, up to 3000 total pair-wise interactions occurred.
Female processing time was set to \(T_{\mathrm{f}}=2\) time steps, and
\(\gamma\) and \(\alpha\) values in the {[}0.2, 2.0{]}, respectively,
were used. The parameter \(\gamma\) was estimated as relative increase
in offspring production such that \(\gamma\) is the factor by which
fitness increases (relative to the baseline fitness) given a nuptial
gift (See Supporting Information S5).}
\end{figure}

\hypertarget{discussion}{%
\section{Discussion}\label{discussion}}

Nuptial gift-giving has arisen several times independently throughout
the animal kingdom (\protect\hyperlink{ref-Lewis2012}{Lewis \& South,
2012}), so understanding how selection favours nuptial gift giving and
choosiness is important for a broad range of mating systems. We provide
a general framework that defines the necessary conditions for selection
to favour the evolution of nuptial gift-giving. We show that males
should give nuptial gifts if the value of a nuptial gift exceeds a
threshold dependent on the encounter rate between males and females and
the cost or time necessary to find or produce a nuptial gift (see ineq.
2). This result makes intuitive sense because if males rarely encounter
females, time searching for a gift is a minor cost relative to mate
search time. If males encounter many females, it is not worth seeking
nuptial gifts unless gifts are very valuable since the male will meet
many prospective mates, and nuptial gift search time might come at a
cost of decreased mating opportunities. In practice, male biased sex
ratios will not necessarily favour male search for nuptial gifts if the
female encounter rate is very high, so the key variable is how often
males and females encounter each other. If the search time or cost of
finding a nuptial gift is high, nuptial gifts must be very valuable
before search is favoured by selection.

\hypertarget{threshold-fitness-values}{%
\subsection{Threshold fitness values}\label{threshold-fitness-values}}

Importantly, we show that the threshold nuptial gift value at which
females are favoured to express choosiness for nuptial gifts is rarely
equivalent to the threshold value at which males are favoured to search
for nuptial gifts, potentially leading to sexual conflict
(\protect\hyperlink{ref-Arnqvist2005a}{Arnqvist \& Rowe, 2005};
\protect\hyperlink{ref-Oliveira2008}{Oliveira \emph{et al.}, 2008}).
Here, we are defining sexual conflict as occurring when interactions
between sexes result in situation where both sexes cannot achieve an
optimal outcome simultaneously
(\protect\hyperlink{ref-Parker2006}{Parker, 2006}). As an example,
sizable areas of parameter space exists wherein the female optimum would
be to exhibit preference for (and receive) nuptial gifts, while the male
optimum is to not search for (and give) nuptial gifts (see Figure 2a,
Zone C). This will lead to mate encounters in which gift-less males will
benefit from mating, but females will not. In many systems, ecological
variables such as search time required to find a nuptial gift will
likely depend on prey abundance, which can vary substantially with time
in some species with nuptial gift-giving
(\protect\hyperlink{ref-Ghislandi2018}{Ghislandi \emph{et al.}, 2018}).
Since several ecological variables likely affect the value of these
thresholds, our results can be seen as providing some formalised
description of why nuptial gift-giving only occurs in some but not all
systems.

At first, the analytical model seems to suggests that nuptial gifts must
cause a very high fitness increase (approximately 25\%) before male
search and female choosiness is favoured by selection (Figure 2).
Similarly the IBM model seems to suggest that a fitness benefit of
approximately 50\% is required (see Figure 3). However, it is important
to recognise that these thresholds depend on multiple parameters. For
example, if female processing time (\(T_{\mathrm{f}}\)) is high, the
female threshold for choosiness with respect to \(\gamma\) drops such
that male search and female choosiness are favoured at lower \(\gamma\)
(see Supporting Information S2). If \(T_{\mathrm{f}}\) is sufficiently
high, then an initially rare gift-giving trait might be favoured by
selection even if the fitness benefit of a nuptial gift is low. The
effect that nuptial gifts have on fitness might vary across species, or
even populations. Effects on female fecundity have been estimated in
crickets, fireflies, butterflies, and spiders, but these estimates vary
considerably between species suggesting a large positive effect to no
effect at all (\protect\hyperlink{ref-Bergstrom2002}{Bergström \&
Wiklund, 2002}; \protect\hyperlink{ref-Rooney2002}{Rooney \& Lewis,
2002}; \protect\hyperlink{ref-Maxwell2018}{Maxwell \& Prokop, 2018};
\protect\hyperlink{ref-Gao2019}{Gao \emph{et al.}, 2019}).

We modelled the evolution of nuptial gift-giving using both a
mathematical model and an individual-based model. Our mathematical model
makes simple assumptions about the relationship between nuptial gift
search time (\(\alpha\)), conspecific encounter rate (\(R\)), female
processing time (\(T_{\mathrm{f}}\)), and the fitness increment of a
nuptial gift for offspring fitness (\(\gamma\)). It then derives the
threshold \(\gamma\) values above which males increase their fitness by
searching (\(T_{\mathrm{m}} > 0\)) for a nuptial gift
(\(\gamma_{\mathrm{m}}\)) and females increase their fitness by choosing
to reject males without gifts (\(\gamma_{\mathrm{f}}\)). In contrast,
our IBM models individuals over discrete time steps, and key processes
of nuptial gift acquisition, conspecific encounters, and female
processing are stochastic and varying among individuals. The
mathematical model and IBM make qualitatively identical predictions
(compare Figure 2b versus Figure 3), but differences between the two
models inevitably lead to quantitative differences for
\(\gamma_{\mathrm{m}}\) and \(\gamma_{\mathrm{f}}\) thresholds. For
example, \(\gamma_{\mathrm{f}}\) increased more rapidly with increasing
\(\alpha\) in the IBM compared to the analytical model. Some differences
are expected to occur due to stochastic effects inherent to IBMs (e.g.,
\protect\hyperlink{ref-Wilson2003}{Wilson \emph{et al.}, 2003}). Other
differences are more likely caused by more subtle assumptions between,
and limitations of, the two models. In general, the IBM did not do a
good job of controlling for conspecific interaction rate, making it
difficult to directly compare \(R\) between models. The IBM also allowed
for coevolution between male search and female choosiness (Figure 3),
which was not allowed in the analytical model (Figure 2). It was not our
goal to exactly recover the quantitative predictions of the analytical
model in our IBM. Future development of the IBM could further bridge the
gap between models while also developing new theory on how aspects of
the system such as explicit space, individual life history, or genetics
affect the evolution of nuptial gift-giving behaviour.

\hypertarget{nuptial-gift-giving-theory}{%
\subsection{Nuptial gift-giving
theory}\label{nuptial-gift-giving-theory}}

When modelling nuptial gift evolution, the challenge is to construct a
framework that captures the frequency-dependent selection between male
nuptial gift-giving and female choosiness for nuptial gifts, and we do
this using a time-in, time-out model. Recent studies have modelled some
frequency-dependent aspect of nuptial gift giving using evolutionary
game theory (\protect\hyperlink{ref-MaynardSmith1982}{Maynard Smith,
1982}; \protect\hyperlink{ref-Vincent2005}{Vincent \& Brown, 2005}). Two
such studies formulated a quantitative genetics model to study
evolutionarily stable nuptial gift sizes in populations where the female
propensity to re-mate was evolving
(\protect\hyperlink{ref-Kamimura2021}{Kamimura \emph{et al.}, 2021};
\protect\hyperlink{ref-Iwasa2022}{Iwasa \& Yamaguchi, 2022}). The
results obtained in these studies complement our results by giving
equilibrium solutions to the evolutionary stable nuptial gift size,
whereas we determine the general conditions under which nuptial
gift-giving will evolve as given by the inequalities we derive.

Other modelling frameworks have made general predictions about sexually
selected traits, and these predictions are not mutually exclusive to
those made by our model. For example, the good genes hypothesis predicts
that costly traits such as nuptial gift-giving can be favoured since
males enduring the cost of a nuptial gift signals to females that their
genes confer high fitness precisely because they can afford this cost
(\protect\hyperlink{ref-Kirkpatrick1996}{Kirkpatrick, 1996};
\protect\hyperlink{ref-Byers2006}{Byers \& Waits, 2006}; but see
\protect\hyperlink{ref-Fromhage2022}{Fromhage \& Henshaw, 2022}). In
other words, costly sexually selected traits are favoured because they
are indicators of overall genetic quality
(\protect\hyperlink{ref-Martinossi-Allibert2019}{Martinossi-Allibert
\emph{et al.}, 2019}). Because of this, nuptial gift-giving could be a
case of condition-dependence where engaging in nuptial gift-giving is
only favourable for male in good condition (e.g., males capable of
successful search (\protect\hyperlink{ref-MaynardSmith1982}{Maynard
Smith, 1982}; \protect\hyperlink{ref-Engqvist2015}{Engqvist \& Taborsky,
2015}; \protect\hyperlink{ref-Ghislandi2018}{Ghislandi \emph{et al.},
2018})). In general, our model demonstrates how nuptial gift-giving
initially evolves before other mechanisms, such as good gene effects,
become relevant.

A nuptial gift can also constitute a dishonest signal of good body
condition since worthless, deceptive nuptial gifts have evolved in
several systems (\protect\hyperlink{ref-LeBas2005}{LeBas \& Hockham,
2005}; \protect\hyperlink{ref-Ghislandi2014}{Ghislandi \emph{et al.},
2014}). This is also the case in \emph{P. mirabilis} where males will
wrap plant parts or an empty exoskeleton in silk, as opposed to an
arthropod prey, and use this as a nuptial gift
(\protect\hyperlink{ref-Albo2011}{Albo \emph{et al.}, 2011};
\protect\hyperlink{ref-Ghislandi2014}{Ghislandi \emph{et al.}, 2014}).
In such systems, worthless nuptial gifts have been shown to reduce the
likelihood that a male is rejected by a female compared to the case
where no nuptial gift is given. However, males offering worthless
nuptial gifts may be at a slight disadvantage in sperm competition since
worthless gifts result in a shorter copulation duration and hence less
sperm transfer (\protect\hyperlink{ref-Albo2013}{Albo \emph{et al.},
2013}; \protect\hyperlink{ref-Ghislandi2014}{Ghislandi \emph{et al.},
2014}). Worthless gifts should not result in any paternal care benefits
to the male since the offspring he may sire will not gain nutrition from
a worthless nuptial gift.

Given our modelling framework, worthless nuptial gifts may be expected
to evolve in cases where females are discriminating in favour of nuptial
gifts, but the cost of search time for a true nuptial gift is very high
such that selection will not favour male search. This scenario would
correspond to zone C of Figure 2 where the value of the nuptial gift
exceeds the female fitness threshold for choosiness to be favoured, but
due to high search time, selection will not favour male search for true
nuptial gifts. Our model also predicts the possibility of the opposite
scenario, in which males provide nuptial gifts, but females do not
exhibit preference for nuptial gifts (zone B of Figure 2).
Fascinatingly, an example of such system has been documented by a recent
study of the genus \emph{Trechaleoides}, which contains two species with
true nuptial gift-giving, but a lack of preference for nuptial gifts
among female (\protect\hyperlink{ref-Martinez2023}{Martínez Villar
\emph{et al.}, 2023}).

The main drivers of male nuptial gift-giving are thought to be indirect
fitness benefits and increased success in sperm competition, since
providing a nuptial gift can result in longer copulation duration which
is correlated with increased sperm transfer along with female cryptic
choice promoting males who provide nuptial gifts
(\protect\hyperlink{ref-Albo2011}{Albo \emph{et al.}, 2011},
\protect\hyperlink{ref-Albo2013}{2013}). However, nuptial gifts might
also function to modulate female aggression and prevent sexual
cannibalism (\protect\hyperlink{ref-Bilde2006}{Bilde \emph{et al.},
2006}). In some systems, such as \emph{P. mirabilis}, males have been
shown to reduce the risk of being cannibalised by the female after
mating when offering a nuptial gift, such that the nuptial gift may
result in a ``shield effect'', protecting the male
(\protect\hyperlink{ref-Toft2016}{Toft \& Albo, 2016}).

\hypertarget{empirical-implications}{%
\subsection{Empirical implications}\label{empirical-implications}}

The simulations parameterised with an experimentally estimated value of
\(\gamma\) showed evolution of nuptial gift searching in males and
choosiness for nuptial gifts in females. The model thus predicts that
\emph{P. mirabilis} living under conditions with the estimated fitness
value of nuptial gifts should exhibit both search for nuptial gifts and
choosiness for males with nuptial gifts, and this is what is observed in
empirical populations. Parameterising \(\gamma\) with data from
experimental studies may only yield a rough approximation of the true
\(\gamma\). This is because the estimated value of \(\gamma\) is based
on data from current populations (rather than ancestral populations,
which are being simulated), and because the literature is inconclusive
as to how much (if any) effect nuptial gifts have on female fitness
(\protect\hyperlink{ref-Maxwell2018}{Maxwell \& Prokop, 2018}). The
effect of nuptial gifts on female fecundity has been estimated in a
variety of system such as crickets, fireflies, butterflies and spiders,
but these estimates vary considerably between species suggesting a large
positive effect to no effect at all
(\protect\hyperlink{ref-Bergstrom2002}{Bergström \& Wiklund, 2002};
\protect\hyperlink{ref-Rooney2002}{Rooney \& Lewis, 2002};
\protect\hyperlink{ref-Maxwell2018}{Maxwell \& Prokop, 2018};
\protect\hyperlink{ref-Gao2019}{Gao \emph{et al.}, 2019}).

Our model assumes something akin to sequential polyandry. That is, a
system wherein female mating and reproduction with multiple males occurs
in sequence, rather than multiple matings occurring before reproduction.
In some systems with nuptial gift-giving, females have been documented
to mate multiple times before reproduction occurs, including the genus
of bark lice \emph{Neotrogla}
(\protect\hyperlink{ref-Kamimura2021}{Kamimura \emph{et al.}, 2021}),
and even our example system of \emph{P. mirabilis} where females will
sometimes engage in multiple mating before reproducing, especially if
starved because multiple mating may result in more nuptial gifts
(\protect\hyperlink{ref-Toft2015}{Toft \& Albo, 2015};
\protect\hyperlink{ref-Matzke2022}{Matzke \emph{et al.}, 2022}). It is
unclear what effect (if any) assuming non-sequential polyandry would
have on the threshold we derive. Under non-sequential polyandry, a
viable strategy for females might be to accept any male (with or without
gift) for fertilisation assurance, then exhibit a preference for nuptial
gifts. This might make choosiness less costly since it would entail less
of an opportunity cost to be choosy, and this could potentially make
female preference for nuptial gifts more likely to evolve. Our model
also assumes that males search in time-out, rather than contribute to
parental care, which is likely to be accurate for most systems but not
all. Expanding the model to explore these possibilities would be a
worthwhile goal for future research.

\hypertarget{conclusion}{%
\subsection{Conclusion}\label{conclusion}}

Overall, we found that a simple relationship between nuptial gift search
time and mate encounter rate yields a threshold that determines whether
selection will favour males that search for nuptial gifts. Similarly, we
found that the threshold determining whether females will be favoured to
reject males without nuptial gifts is also dependent on these variables,
along with offspring processing time. Together, these thresholds
describe the conditions under which nuptial gift-giving is expected to
evolve. The applications of these thresholds are numerous. They can be
used as a starting point for more complex or more system-specific models
of nuptial gift-giving evolution. They can also provide novel insight
into how populations can evolve to use worthless or token nuptial gifts.

\hypertarget{methods}{%
\section{Methods}\label{methods}}

\hypertarget{model}{%
\subsection{Model}\label{model}}

Here we first present more detail for the derivation of fitness
threshold values \(\gamma_{\mathrm{m}}\) and \(\gamma_{\mathrm{f}}\). We
then present IBM simulations (see Supporting Information S1 for full
details). Code for simulations is available on GitHub (see ``Data
availability'').

\hypertarget{derivation-of-fitness-thresholds}{%
\subsection{Derivation of fitness
thresholds}\label{derivation-of-fitness-thresholds}}

We use a time-in and time-out model in which females and males spend
some time searching for a mate (time-in) followed by a period of cool
down outside the mating pool (time-out; Figure 1).

After mating, females must spend some time processing offspring
(\(T_{\mathrm{f}}\)). Male time to replenish sperm is assumed to be
negligible, but males can spend time out of the mating pool
(\(T_{\mathrm{m}}\)) to search for a nuptial gift. When males return
from time-out, they encounter females with some probability that is a
function of the rate at which an individual encounters conspecifics
(\(R\)) and the sex ratio (\(\beta\); males/females). Mortality occurs
for females and males in (\(\mu_{\mathrm{i,f}}\),
\(\mu_{\mathrm{i,m}}\)) and out (\(\mu_{\mathrm{o,f}}\),
\(\mu_{\mathrm{o,m}}\)) of the mating pool. Following Kokko \& Ots
(\protect\hyperlink{ref-Kokko2006}{2006}), we assume
\(m_{\mathrm{i,f}} = m_{\mathrm{o,f}} = 1\) and
\(m_{\mathrm{i,m}} = m_{\mathrm{o,m}} = 1\). While this choice is
arbitrary, we conducted a sensitivity analysis which shows that the
mortality parameters have no influence on the propensity for male search
and female choice to evolve (Supporting Information S2). First, we
describe the fitness consequences of male search time for a nuptial
gift. We then describe the fitness consequences of female choice to
accept or reject males based on their provision of a nuptial gift.

\hypertarget{male-fitness}{%
\subsection{Male fitness}\label{male-fitness}}

During time-out, males have the opportunity to search for a nuptial
gift. Males can adopt one of two strategies; either search or do not
search for a nuptial gift. Males with the former strategy continue to
search until they find a nuptial gift, while males that do not search
will immediately re-enter the mating pool. In this case, time searching
for a nuptial gift will come at the cost of mating opportunities but
might increase offspring fitness. We therefore need to model the
expected length of time \(E[T_{\mathrm{m}}]\) spent outside of the
mating pool for males that search for nuptial gifts, which is simply
\(\alpha\). Note that we can integrate search time \(t\) over the rate
at which nuptial gifts are encountered (\(\exp(-1/\alpha)\)) to show
\(E[T_{\mathrm{m}}] = \alpha\),

\[E[T_{\mathrm{m}}] = \int_{0}^{\infty}e^{- \frac{1}{\alpha}t}dt = \alpha.\]

The rate at which a focal male that searches for a nuptial gift
(\(\mathrm{M_{G}}\)) increases his fitness is therefore the fitness of
his offspring \((1 + \gamma)\) divided by expected time spent searching
for a nuptial gift (\(\alpha\)) plus time spent in the mating pool,
\((\beta + 1)/R\) (recall that females produce \(\lambda\) offspring),

\[W_{\mathrm{M_{G}}} = \lambda \frac{1 + \gamma}{\alpha + \left( \frac{\beta + 1}{R} \right)}.\]

In contrast, a male that does not search for a nuptial gift
(\(\mathrm{M_{L}}\)) has offspring with lower fitness, but spends less
time outside of the mating pool,

\[W_{\mathrm{M_{L}}} = \lambda \frac{1}{\left(\frac{\beta+1}{R} \right)} = \lambda \frac{R}{\beta + 1}.\]

We can then determine the conditions for which
\(W_{\mathrm{M_{G}}} > W_{\mathrm{M_{L}}}\), isolating \(\gamma\) to
find how large of a fitness benefit must be provided by the nuptial gift
to make the search cost worthwhile, which simplifies to ineq. 2. When
this inequality holds, males are favoured to search until they find a
nuptial gift, which would result in an average search time of
\(\alpha\). When the male trait is continuous (i.e., males search for
time period \(T_{\mathrm{m}}\)), it can be shown that the same threshold
can be reached by evaluating the partial derivative of the male fitness
function (Supporting Information S3). Hence, the thresholds are
consistent under different assumptions concerning male searching
strategy. Selection will cause males to search for nuptial gifts if the
fitness increase to offspring exceeds the product of search time and
female encounter rate.

\hypertarget{female-fitness}{%
\subsection{Female fitness}\label{female-fitness}}

During time-out, females process offspring over a duration of
\(T_{\mathrm{f}}\) (we assume that \(T_{\mathrm{f}} > \alpha\), else
females are not the choosy sex). When females re-enter the mating pool,
they encounter males at a rate of \(R\beta/(\beta + 1)\). If a female
encounters a male with a nuptial gift, we assume that she will mate with
him. But if a female encounters a male with no nuptial gift, then she
might accept or reject the male. If she rejects the male, then she will
remain in the mating pool. The rate at which a female encounters a male
with a nuptial gift is,

\[R_{\mathrm{F_{G}}} = R \left(\frac{\beta}{\beta + 1}\right)\left(1 - e^{-\frac{1}{\alpha}T_{m}}\right).\]

We can similarly model the rate at which a female encounters a gift-less
male,

\[R_{\mathrm{F_{L}}} = R \left(\frac{\beta}{\beta + 1}\right)\left(e^{-\frac{1}{\alpha}T_{m}}\right).\]

Note that we can recover the rate at which a female encounters any male,

\[R \left(\frac{\beta}{\beta + 1}\right) = R \left(\frac{\beta}{\beta + 1}\right) \left(1 - e^{-\frac{1}{\alpha}T_{\mathrm{m}}}\right) + R \left(\frac{\beta}{\beta + 1}\right) \left(e^{-\frac{1}{\alpha}T_{\mathrm{m}}}\right).\]

If \(R_{\mathrm{F_{G}}}\) is sufficiently high and
\(R_{\mathrm{F_{L}}}\) is sufficiently low, then finding a male with a
gift will be easier than finding a male without one. Also, the expected
time spent in the mating pool before a focal female encounters a male
with a gift will be \(1/R_{\mathrm{F_{G}}}\), while the expected time
spent in the mating pool before a focal female encounters any male will
be \(1 / (R_{\mathrm{F_{G}}} + R_{\mathrm{F_{L}}})\). Finally, the rates
at which a female encounters males with and without a gift,
\(R_{\mathrm{F_{G}}}\) and \(R_{\mathrm{F_{L}}}\), are different from
the probabilities that a male encounter has or does not have a gift. The
rate of encounter is no longer relevant in this case because we are
assuming that an encounter has occurred. Hence, the probability of an
encountered male having a gift is simply,

\[P(G) = \frac{1 - e^{-\frac{1}{\alpha}T_{\mathrm{m}}}}{\left(1 - e^{-\frac{1}{\alpha}T_{\mathrm{m}}}\right) + e^{-\frac{1}{\alpha}T_{\mathrm{m}}}} = 1 - e^{-\frac{1}{\alpha}T_{\mathrm{m}}}.\]

Similarly, the probability of an encountered male not having a gift is
simply,

\[P(L) = e^{-\frac{1}{\alpha}T_{\mathrm{m}}}.\]

The rate at which a female increases her fitness by being choosy and
mating only when she encounters a male with a gift is,

\begin{equation}
W_{\mathrm{F_{G}}} = \lambda \frac{1 + \gamma}{T_{\mathrm{f}}} + \frac{1}{\mathrm{F_{G}}}.
\end{equation}

The top of the right-hand side of Eq. 6 gives the fitness increase, and
the bottom gives the total time it takes to obtain this fitness. The
\(R_{\mathrm{F_{G}}}\) is inverted because it represents the expected
time to encountering a male with a gift. We can expand Eq. 6,

\[W_{\mathrm{F_{G}}} = \lambda \frac{1 + \gamma}{T_{\mathrm{f}} + \frac{1}{R \left(\frac{\beta}{\beta + 1}\right)\left(1 - e^{-\frac{1}{\alpha}T_{\mathrm{m}}}\right)}}.\]

If the focal female is not choosy and accepts the first male that she
encounters, then the rate at which she increases her fitness is,

\[W_{\mathrm{F_{G,L}}} = \lambda \frac{\left(1 + \gamma\right)\left(1 - e^{-\frac{1}{\alpha}T_{\mathrm{m}}}\right) + e^{-\frac{1}{\alpha}T_{\mathrm{m}}}}{T_{f} + \frac{1}{R \left(\frac{\beta}{\beta + 1}\right)}}.\]

We then evaluate the conditions under which
\(W_{\mathrm{F_{G}}} > W_{\mathrm{F_{G,L}}}\). We isolate \(\gamma\) to
determine how much offspring fitness must be increased to make
choosiness beneficial (\(\gamma_{\mathrm{f}}\)),

\[\gamma_{\mathrm{f}} > \frac{1 + \frac{1}{\beta}}{D T_{f} \left(1 - e^{-1\frac{1}{\alpha}T_{m}}\right)}.\]

The above can be expressed as Eq. 7 below,

\begin{equation}
\gamma_{\mathrm{f}} > \frac{1}{T_{\mathrm{f}}} R\left(\frac{\beta}{\beta + 1}\right) \left(1 - e^{-1\frac{1}{\alpha}T_{\mathrm{m}}}\right).
\end{equation}

Note that that the expression
\(R\left(\frac{\beta}{\beta + 1}\right) \left(1 - e^{-1\frac{1}{\alpha}T_{\mathrm{m}}}\right)\)
defines the rate at which a female in the mating pool encounters males
with nuptial gifts. Hence, female choosiness is ultimately determined by
time spent out of the mating pool to process offspring
(\(T_{\mathrm{f}}\)) and the rate at which a female in the mating pool
encounters males with nuptial gifts.

\hypertarget{operational-sex-ratio}{%
\subsection{Operational sex ratio}\label{operational-sex-ratio}}

We assume that the sex ratio is equal upon maturation. Given this, Kokko
\& Monaghan (\protect\hyperlink{ref-Kokko2001}{2001}) show that the
operational sex ratio depends on the probability of finding an
individual in time-in,

\begin{equation}
\beta = \frac{\int_{t = 0}^{\infty} P_{IM}(t)dt}{\int_{t = 0}^{\infty} P_{IF}(t)dt}.
\end{equation}

In Eq. 8, \(P_{IM}(t)\) and \(P_{IF}(t)\) are the probabilities of
finding a male and female in time-in, respectively. There is no closed
form solution to the operational sex ratio, so we used recursion to
calculate \(\beta\) values for a given \(T_{\mathrm{f}}\),
\(T_{\mathrm{m}}\), and \(R\) (see Supporting Information S4),

\hypertarget{individual-based-model}{%
\subsection{Individual-based model}\label{individual-based-model}}

We formulate an individual-based simulation model to test whether the
predictions made by the analytical time-in time-out model are
qualitatively the same under a similar simulation model. We use the
individual-based model to test whether the prediction hold in finite
populations. The IBM was written in C. All details of the
initialisation, parameterisation and the specific simulations which we
run, are described in Supporting Information S1.

We use available experimental data on the effect of nuptial gifts on
female offspring production to estimate the key parameter \(\gamma\)
(fitness increment from nuptial gift) and conducted a series of
simulations where \(\gamma\) was parameterised using this estimated
value. Details of the estimation of \(\gamma\) are described in
Supporting Information S5. Separate evolution of male search and female
choice (i.e., without co-evolution) is simulated in Supporting
Information S6.

\textbf{Author contributions:} APC and ABD conceived the study. ABD
constructed the modelling framework with input from APC. APC wrote the
paper with input from ABD and ABD wrote the IBM model. TB and GB
provided substantial comments on previous drafts and final text.

\textbf{Acknowledgements:} Anders P. Charmouh was supported by the
University of Aberdeen. A. Bradley Duthie was supported by the
University of Stirling. Trine Bilde was supported by The Danish Council
for Independent Research grant number 4002-00328B. We thank Maria Albo
for making available data from a previous study which was useful for
parameter estimating.

\textbf{Data availability:} The simulation software was implemented in C
and the full source code is available at
\url{https://github.com/bradduthie/nuptial_gift_evolution}.

\textbf{Competing interests:} The authors declare no competing
interests.

\textbf{References}

\hypertarget{refs}{}
\begin{CSLReferences}{0}{0}
\leavevmode\vadjust pre{\hypertarget{ref-Albo2013}{}}%
Albo, M.J., Bilde, T. \& Uhl, G. (2013)
\href{https://doi.org/10.1098/rspb.2013.1735}{{Sperm storage mediated by
cryptic female choice for nuptial gifts}}. \emph{Proceedings of the
Royal Society B: Biological Sciences}, \textbf{280}.

\leavevmode\vadjust pre{\hypertarget{ref-Albo2019}{}}%
Albo, M.J., Franco-Trecu, V., Wojciechowski, F.J., Toft, S. \& Bilde, T.
(2019) \href{https://doi.org/10.1093/beheco/arz040}{{Maintenance of
deceptive gifts in a natural spider population: Ecological and
demographic factors}}. \emph{Behavioral Ecology}, \textbf{30},
993--1000.

\leavevmode\vadjust pre{\hypertarget{ref-Albo2014}{}}%
Albo, M.J., Toft, S. \& Bilde, T. (2014) {Sexual selection, ecology, and
evolution of nuptial gifts in spiders}. In \emph{Sexual selection:
Perspectives and models from the neotropics} (ed. by Macedo, R.H. \&
Machado, G.). Academic Press, London, pp. 183--200.

\leavevmode\vadjust pre{\hypertarget{ref-Albo2011}{}}%
Albo, M.J., Winther, G., Tuni, C., Toft, S. \& Bilde, T. (2011)
\href{https://doi.org/10.1186/1471-2148-11-329}{{Worthless donations:
Male deception and female counter play in a nuptial gift-giving
spider}}. \emph{BMC Evolutionary Biology}, \textbf{11}.

\leavevmode\vadjust pre{\hypertarget{ref-Arnqvist2005a}{}}%
Arnqvist, G. \& Rowe, L. (2005) \emph{{Sexual Conflict}}. Princeton
University Press, Princeton, New Jersey.

\leavevmode\vadjust pre{\hypertarget{ref-Bergstrom2002}{}}%
Bergström, J. \& Wiklund, C. (2002) Effects of size and nuptial gifts on
butterfly reproduction: Can females compensate for a smaller size
through male-derived nutrients? \emph{Behavioral Ecology and
Sociobiology}, \textbf{52}, 296--302.

\leavevmode\vadjust pre{\hypertarget{ref-Bilde2006}{}}%
Bilde, T., Tuni, C., Elsayed, R., Pekár, S. \& Toft, S. (2006)
\href{https://doi.org/10.1098/rsbl.2005.0392}{{Death feigning in the
face of sexual cannibalism}}. \emph{Biology Letters}, \textbf{2},
23--25.

\leavevmode\vadjust pre{\hypertarget{ref-Bristowe1926}{}}%
Bristowe, W. \& Locket, G.H. (1926) {The courtship of British lycosid
spiders, and its probable significance}. \emph{Proceedings of the
Zoological Society of London}, \textbf{96}, 317--347.

\leavevmode\vadjust pre{\hypertarget{ref-Byers2006}{}}%
Byers, J.A. \& Waits, L. (2006)
\href{https://doi.org/10.1073/pnas.0608184103}{{Good genes sexual
selection in nature}}. \emph{Proceedings of the National Academy of
Sciences}, \textbf{103}, 16343--16345.

\leavevmode\vadjust pre{\hypertarget{ref-Clutton-Brock1992}{}}%
Clutton-Brock, T.H. \& Parker, G.A. (1992)
\href{https://doi.org/10.1086/417793}{{Potential reproductive rates and
the operation of sexual selection}}. \emph{Quarterly Review of Biology},
\textbf{67}, 437--456.

\leavevmode\vadjust pre{\hypertarget{ref-Engqvist2015}{}}%
Engqvist, L. \& Taborsky, M. (2015)
\href{https://doi.org/10.1098/rspb.2015.2945}{{The evolution of genetic
and conditional alternative reproductive tactics}}. \emph{Proceedings of
the Royal Society B: Biological Sciences}, \textbf{283}.

\leavevmode\vadjust pre{\hypertarget{ref-Fromhage2022}{}}%
Fromhage, L. \& Henshaw, J.M. (2022) The balance model of honest sexual
signaling. \emph{Evolution}, \textbf{76}, 445--454.

\leavevmode\vadjust pre{\hypertarget{ref-Gao2019}{}}%
Gao, Q., Turnell, B.R., Hua, B. \& Shaw, K.L. (2019) The effect of
nuptial gift number on fertilization success in a hawaiian swordtail
cricket. \emph{Behavioral Ecology and Sociobiology}, \textbf{73}, 1--9.

\leavevmode\vadjust pre{\hypertarget{ref-Ghislandi2014}{}}%
Ghislandi, P.G., Albo, M.J., Tuni, C. \& Bilden, T. (2014)
\href{https://www.ptonline.com/articles/how-to-get-better-mfi-results}{{Evolution
of deceit by worthless donations in a nuptial gift-giving spider}}.
\emph{Current Zoology}, \textbf{60}, 43--51.

\leavevmode\vadjust pre{\hypertarget{ref-Ghislandi2018}{}}%
Ghislandi, P.G., Pekár, S., Matzke, M., Schulte-Döinghaus, S., Bilde, T.
\& Tuni, C. (2018) \href{https://doi.org/10.1111/jeb.13284}{{Resource
availability, mating opportunity and sexual selection intensity
influence the expression of male alternative reproductive tactics}}.
\emph{Journal of Evolutionary Biology}, \textbf{31}, 1035--1046.

\leavevmode\vadjust pre{\hypertarget{ref-Iwasa2022}{}}%
Iwasa, Y. \& Yamaguchi, S. (2022)
\href{https://doi.org/10.1016/j.jtbi.2021.110939}{{Evolution of male
nuptial gift and female remating: A quantitative genetic model}}.
\emph{Journal of Theoretical Biology}, \textbf{533}, 110939.

\leavevmode\vadjust pre{\hypertarget{ref-Kamimura2021}{}}%
Kamimura, Y., Yoshizawa, K., Lienhard, C., Ferreira, R.L. \& Abe, J.
(2021) \href{https://doi.org/10.1186/s12862-021-01901-x}{{Evolution of
nuptial gifts and its coevolutionary dynamics with male-like persistence
traits of females for multiple mating}}. \emph{BMC Ecology and
Evolution}, \textbf{21}, 1--14.

\leavevmode\vadjust pre{\hypertarget{ref-Kirkpatrick1996}{}}%
Kirkpatrick, M. (1996) Good genes and direct selection in the evolution
of mating preferences. \emph{Evolution}, \textbf{50}, 2125--2140.

\leavevmode\vadjust pre{\hypertarget{ref-Kokko2001}{}}%
Kokko, H. \& Monaghan, P. (2001)
\href{https://doi.org/10.1046/j.1461-0248.2001.00212.x}{{Predicting the
direction of sexual selection}}. \emph{Ecology Letters}, \textbf{4},
159--165.

\leavevmode\vadjust pre{\hypertarget{ref-Kokko2006}{}}%
Kokko, H. \& Ots, I. (2006) {When not to avoid inbreeding}.
\emph{Evolution}, \textbf{60}, 467--475.

\leavevmode\vadjust pre{\hypertarget{ref-LeBas2005}{}}%
LeBas, N.R. \& Hockham, L.R. (2005) \href{https://doi.org/10.1016/j}{{An
invasion of cheats: the evolution of worthless nuptial gifts}}.
\emph{Current Biology}, \textbf{15}, 64--67.

\leavevmode\vadjust pre{\hypertarget{ref-Lewis2014}{}}%
Lewis, S.M., Vahed, K., Koene, J.M., Engqvist, L., Bussière, L.F.,
Perry, J.C., \emph{et al.} (2014)
\href{https://doi.org/10.1098/rsbl.2014.0336}{{Emerging issues in the
evolution of animal nuptial gifts}}. \emph{Biology Letters},
\textbf{10}, 3--5.

\leavevmode\vadjust pre{\hypertarget{ref-Lewis2012}{}}%
Lewis, S. \& South, A. (2012)
\emph{\href{https://doi.org/10.1016/B978-0-12-394288-3.00002-2}{{The
Evolution of Animal Nuptial Gifts}}}.

\leavevmode\vadjust pre{\hypertarget{ref-Lubin2007}{}}%
Lubin, Y. \& Bilde, T. (2007)
\href{https://doi.org/10.1016/S0065-3454(07)37003-4}{{The evolution of
sociality in spiders}}. \emph{Advances in the Study of Behavior},
\textbf{37}, 83--145.

\leavevmode\vadjust pre{\hypertarget{ref-Martinez2023}{}}%
Martínez Villar, M., Germil, M., Pavón-Peláez, C., Tomasco, I., Bilde,
T., Toft, S., \emph{et al.} (2023) Lack of female preference for nuptial
gifts may have led to loss of the male sexual trait. \emph{Evolutionary
Biology}, \textbf{50}, 318--331.

\leavevmode\vadjust pre{\hypertarget{ref-Martinossi-Allibert2019}{}}%
Martinossi-Allibert, I., Rueffler, C., Arnqvist, G. \& Berger, D. (2019)
\href{https://doi.org/10.1098/rspb.2018.2313}{The efficacy of good genes
sexual selection under environmental change}. \emph{Proceedings of the
Royal Society B: Biological Sciences}, \textbf{286}.

\leavevmode\vadjust pre{\hypertarget{ref-Matzke2022}{}}%
Matzke, M., Toft, S., Bechsgaard, J., Pold Vilstrup, A., Uhl, G.,
Künzel, S., \emph{et al.} (2022) Sperm competition intensity affects
sperm precedence patterns in a polyandrous gift-giving spider.
\emph{Molecular Ecology}, \textbf{31}, 2435--2452.

\leavevmode\vadjust pre{\hypertarget{ref-Maxwell2018}{}}%
Maxwell, M.R. \& Prokop, P. (2018) Fitness effects of nuptial gifts in
the spider \emph{pisaura mirabilis}: Examination under an alternative
feeding regime. \emph{The Journal of Arachnology}, \textbf{46},
404--412.

\leavevmode\vadjust pre{\hypertarget{ref-MaynardSmith1982}{}}%
Maynard Smith, J. (1982)
\emph{\href{https://www.ncbi.nlm.nih.gov/pubmed/13761767}{{Evolution and
the theory of games}}}. Cambridge University Press, Cambridge.

\leavevmode\vadjust pre{\hypertarget{ref-Oliveira2008}{}}%
Oliveira, R.F., Taborsky, M. \& Brockmann, H.J. (Eds.). (2008)
\emph{{Alternative reproductive tactics: an integrative approach}}.
Cambridge University Press, Cambridge.

\leavevmode\vadjust pre{\hypertarget{ref-Parker2006}{}}%
Parker, G.A. (2006) \href{https://doi.org/10.1098/rstb.2005.1785}{Sexual
conflict over mating and fertilization: An overview}.
\emph{Philosophical Transactions of the Royal Society B}, \textbf{361},
235--259.

\leavevmode\vadjust pre{\hypertarget{ref-Rooney2002}{}}%
Rooney, J. \& Lewis, S.M. (2002) Fitness advantage from nuptial gifts in
female fireflies. \emph{Ecological Entomology}, \textbf{27}, 373--377.

\leavevmode\vadjust pre{\hypertarget{ref-Toft2015}{}}%
Toft, S. \& Albo, M.J. (2015) Optimal numbers of matings: The
conditional balance between benefits and costs of mating for females of
a nuptial gift-giving spider. \emph{Journal of Evolutionary Biology},
\textbf{28}, 457--467.

\leavevmode\vadjust pre{\hypertarget{ref-Toft2016}{}}%
Toft, S. \& Albo, M.J. (2016)
\href{https://doi.org/10.1098/rsbl.2015.1082}{{The shield effect:
Nuptial gifts protect males against pre-copulatory sexual cannibalism}}.
\emph{Biology Letters}, \textbf{12}.

\leavevmode\vadjust pre{\hypertarget{ref-Tuni2013a}{}}%
Tuni, C., Albo, M.J. \& Bilde, T. (2013)
\href{https://doi.org/10.1111/jeb.12137}{{Polyandrous females acquire
indirect benefits in a nuptial feeding species}}. \emph{Journal of
Evolutionary Biology}, \textbf{26}, 1307--1316.

\leavevmode\vadjust pre{\hypertarget{ref-Vincent2005}{}}%
Vincent, T.L. \& Brown, J.S. (2005) \emph{{Evolutionary Game Theory,
Natural Selection, and Darwinian Dynamics}}. Cambridge University Press,
Cambridge.

\leavevmode\vadjust pre{\hypertarget{ref-Wilson2003}{}}%
Wilson, W.G., Morris, W.F. \& Bronstein, J.L. (2003) {Coexistence of
mutualists and exploiters on spatial landscapes}. \emph{Ecological
Monographs}, \textbf{73}, 397--413.

\end{CSLReferences}

\clearpage

\hypertarget{supporting-information}{%
\section{Supporting Information}\label{supporting-information}}

\begin{longtable}[]{@{}ll@{}}
\toprule
Information & Page \\
\midrule
\endhead
S1: Description of the individual-based simulation model & 18 \\
S2: Sensitivity analysis of mortality parameters in IBM & 20 \\
S3: Alternative derivation of male fitness threshold & 29 \\
S4: Operational sex ratio & 32 \\
S5: Estimation of key model parameter using experimental data & 33 \\
S6: Separate evolution of male search and female choice & 34 \\
\bottomrule
\end{longtable}

\clearpage

\hypertarget{s1-description-of-the-individual-based-simulation-model}{%
\subsection{S1: Description of the individual-based simulation
model}\label{s1-description-of-the-individual-based-simulation-model}}

Here we describe the details of initialisation, time-in (mating),
time-out (reproduction and nuptial gift search), and mortality. We then
summarise the simulations run and data collected.

\emph{Initialisation}

Before the first time step, a population of \(N = 1000\) individuals is
initialised. Individuals are assigned unique IDs, and each is assigned
to be female with a probability of \(0.5\), else male. Each individual
\(i\) is initialised with a starting value of female offspring
processing time (\(T^{i}_{\mathrm{f}}\)), rejection probability of
gift-less males (\(\rho^{i}\)), and male search time
(\(T^{i}_{\mathrm{m}}\)). For all simulations, initialised values are
set to \(T^{i}_{\mathrm{f}} = 2\), \(\rho^{i} = 0\), and
\(T^{i}_{\mathrm{m}} = 0\). All individuals are initalised outside of
the mating pool in the first time step \(t = 1\). The first time step
then proceeds with females immediately entering the mating pool and
males either entering the mating pool or searching for nuptial gifts.

\emph{Time-in}

At the start of each time step, females and males in the mating pool
remain in it. Females will enter the mating pool after processing
offspring, and males will enter it after searching for nuptial gifts
(see `Time-out' below). Up to \(\Psi = N\psi\) interactions between
individuals can occur in a single time step, where \(N\) is population
size and \(\psi\) is a scaling parameter. In each time step, \(\Psi\)
pairs of individuals are selected at random to interact. For each
interaction, two different individuals are randomly selected from the
population with equal probability. If the selected individuals are of
different sexes, and both are in the mating pool, then a mating
encounter occurs. If the male does not have a nuptial gift, then the
female will reject him with a probability of \(\rho^{i}\); if rejection
occurs, then both individuals stay in the mating pool. If rejection does
not occur, or the male has a nuptial gift in the mating encounter, then
the individuals mate. Females leave the mating pool and enter time-out
to process offspring, and males leave and enter time-out to potentially
search for new nuptial gifts (note that females and males might re-enter
the mating pool immediately within the same time step given sufficiently
low search time; see Time-out below).

\emph{Time-out}

During time-out, offspring production and time outside of the mating
pool are realised for each female by sampling from a Poisson
distribution. A focal female \(i\) will produce \(Poisson(\lambda)\)
offspring if no nuptial gift was provided or
\(Poisson(\lambda + \gamma)\) if a gift was provided. Females remain
outside of the mating pool to process offspring for
\(Poisson(T^{i}_{\mathrm{f}})\) time steps. Offspring are added to the
population immediately, with \(\rho^{i}\) and \(T^{i}_{\mathrm{m}}\)
values that are the average of each parent plus some normally
distributed error \(\epsilon_{R}\) and \(\epsilon_{T_{\mathrm{m}}}\).
For example,

\[T^{\mathrm{offspring}}_{\mathrm{m}} \sim \frac{T^{\mathrm{mother}}_{\mathrm{m}}  + T^{\mathrm{father}}_{\mathrm{m}}}{2} + N\left(0, \epsilon_{T_{\mathrm{m}}} \right).\]

The variation generated by \(\epsilon\) values simulates mutation upon
which selection for traits can act. In all simulations, \(\epsilon = 0\)
if a trait is fixed and \(\epsilon = 0.01\) if the trait evolves.
Offspring sex is randomly assigned with equal probability as female or
male. Female offspring are immediately placed in the mating pool, and
male offspring are out of the mating pool to potentially search for
nuptial gifts. After a female has spent \(T^{i}_{\mathrm{f}}\) time
steps outside the mating pool, she will re-enter it.

A focal male \(i\) outside the mating pool will enter it if he has
searched for a fixed number of \(T^{i}_{\mathrm{m}}\) time steps, which
is also sampled randomly from a Poisson distribution,
\(Poisson(T^{i}_{\mathrm{m}})\). If \(T^{i}_{\mathrm{m}} = 0\), then the
male immediately returns to the mating pool (in the same time step). If
\(T^{i}_{\mathrm{m}} > 0\), then the male must wait outside the mating
pool for \(Poisson(T^{i}_{\mathrm{m}})\) time steps, but will enter the
mating pool with a nuptial gift with a probability,

\[P(G^{i}) = 1 - e^{-\frac{1}{\alpha}T^{i}_{\mathrm{m}}}.\]

Males must always spend \(T^{i}_{\mathrm{m}}\) time steps outside of the
mating pool regardless of whether or not they are successful in
obtaining a nuptial gift.

\emph{Mortality}

At the end of each time step, mortality occurs first with a fixed
probability for all adults in the population, then with a probability
caused by carrying capacity \(K\) applied to all individuals (adults and
offspring). Mortality occurs in each time step with a fixed probability
of \(\mu\) regardless of the sex of the individual or its position in or
out of the mating pool. If after this fixed mortality is applied, the
total population size \(N > K\), then individuals are removed at random
with equal probability until \(N = K\). Following adult mortality, a new
time step begins with newly added offspring becoming adults.

\emph{Simulations}

We ran simulations in which male search time and female choosiness
evolved from an ancestral state of no searching and no choosiness. In
all simulations, \(N\) was initialised at 1000 and \(K = 1000\).
Simulations ran for \(t_{max} = 40000\) time steps. We set
\(T_{\mathrm{f}} = 2\), \(\psi = 3\), and \(\lambda = 1\) for all
simulations, and we simulated across a range of
\(\alpha = \{0.1, 0.2, ..., 1.9, 2.0\}\) and
\(\gamma = \{0, 0.1, ..., 1.9, 2.0\}\) parameter values for 3200
replicates. Summary statistics for mean trait values, population size,
sex ratios, proportion of females and males in and out of the mating
pool, and mean number of encounters per female and male within the
mating pool were all calculated in the last time step. The C code used
for simulating these IBMs also allows for the reporting of statistics in
each time step. Additionally, it can simulate explicit space and
individual movement through the landscape. A neutral evolving trait was
also modelled to ensure that the code functioned as intended, and
processes were compartmentalised into individual functions to facilitate
code testing. All code is publicly available on GitHub
(\url{https://github.com/bradduthie/nuptial_gift_evolution}).

A set of simulations with a value of \(\gamma\) calculated from
empirical data was also conducted (100 replicates). Here, \(\gamma\) was
parameterised using data on egg production as a function of eating
nuptial gifts (see below). Additional simulation sets with lower and
upper bounds of the estimated \(\gamma\) were subsequently run. In the
simulation sets with experimentally derived parameter values, all other
parameters were identical to previous simulation batches.

We can produce an estimate of the fitness increment obtained by females
when receiving a gift (\(\hat{\gamma}\)) by using data on female
\emph{P. mirabilis} egg production and hatching success under different
feeding regimes from (\protect\hyperlink{ref-Tuni2013a}{Tuni \emph{et
al.}, 2013}). Tuni \emph{et al.}
(\protect\hyperlink{ref-Tuni2013a}{2013}) found differences in egg
production and hatching success in female \emph{P. mirabilis} under
different feeding regimes. Assuming these differences in feeding regimes
correspond to eating versus not eating nuptial gifts, the mean number of
offspring produced by a female who eats nuptial gifts can be calculated
(Table 1).

\clearpage

\hypertarget{s2-sensitivity-analysis-of-parameters-in-ibm}{%
\subsection{S2: Sensitivity analysis of parameters in
IBM}\label{s2-sensitivity-analysis-of-parameters-in-ibm}}

We investigated the sensitivity of our results to individual mortality
during time-in (\(\mu_{\mathrm{in}}\)) and time-out
(\(\mu_{\mathrm{out}}\)), female processing time (\(T_{\mathrm{f}}\)),
potential for interactions between conspecifics (\(\psi\)), and
initialised male search time (\(T_{\mathrm{m}}\)). Across all of these
simulations, there were challenges with statistical power. Evidence for
the evolution of male search (blue points in figures) and female
choosiness (red points in figures) was determined by the lower 95\%
bootstrapped confidence interval of \(T_{\mathrm{m}}\) and
\(T_{\mathrm{f}}\) values being greater than zero, respectively. This
required a lot of replicate simulations in the main text (Figure 3),
especially for values just above predicted thresholds and for female
choosiness. Computation time was a limiting factor, even using a
compiled language (C) and with access to a computing cluster. Absence of
points above threshold values are not necessarily evidence that
evolution of male search or female choosiness is not predicted to evolve
in these regions of parameter space, but it does indicate that evolution
of these traits is not necessarily assured given the stochasticity
inherent to the IBM. Additional simulations can be conducted using the C
code in the `src' folder of the GitHub repository
(\url{https://github.com/bradduthie/nuptial_gift_evolution}). Below, we
explain the parameter values used in the sensitivity analysis in more
detail.

\emph{Mortality}

We conducted a sensitivity analysis of the effect of the mortality
parameters \(\mu_{\mathrm{in}}\) and \(\mu_{\mathrm{out}}\) (the
probability of mortality in time-in and time-out, respectively, which we
assumed to be equal for all individuals) on the evolution of male search
and female choice using the IBM (See S1). The results revealed no
correlation between the value of the mortality parameters and the
evolution of male search or female choice (Fig. S2.1).

\emph{Female processing time}

We also conducted a sensitivity analysis of female processing time
\(T_{\mathrm{f}}\). To do this, we ran simulations at default values,
but with \(T_{\mathrm{f}} = 0.4\) (Figure S4.2) and
\(T_{\mathrm{f}} = 10.0\) (Figures S4.3).

\emph{Interactions between conspecifics}

We conducted a sensitivity analysis on the encounter rate between
conspecifics (\(R\)) by varying the value of \(\psi\) (see S1). Under
default simulations, \(\psi = 3\). We also ran simulations in which
\(\psi = 1\) (Figure S4.4), \(\psi = 2\) (Figure S4.5), \(\psi = 4\)
(Figure S4.6), and \(\psi = 6\) (Figure S4.7), with all other parameters
being set to default values.

\emph{Initialised male search time}

Lastly, we conducted a sensitivity analysis on initialised male search
time (\(T_{\mathrm{m}}\)). To do this, we ran simulations at default
values, but initialised all males at \(T_{\mathrm{m}} = 0.5\) (Figure
S4.8) or \(T_{\mathrm{m}} = 2.5\) (S4.9). Note that under these
initialisation conditions, it no longer makes sense to test whether or
not \(T_{\mathrm{m}}\) evolves to positive values, so we do not show
predicted male search thresholds and focus only on female choosiness.
Interestingly, while predictions given initialised
\(T_{\mathrm{m}} = 0.5\) are quite close for female choosiness
thresholds, female choosiness does not appear to evolve given
\(T_{\mathrm{m}} = 2.5\). This might be due to low selection pressure to
reject males without gifts due to their low frequency. In other words,
while females would benefit from a strategy of rejecting gift-less
males, a lack of realised encounters with gift-less males preclude the
evolution of sustained selection for choosiness.

\clearpage

\captionsetup{labelformat=empty}

\begin{figure}
\centering
\includegraphics{ms_biorxiv_files/figure-latex/unnamed-chunk-6-1.pdf}
\caption{Figure S2.1: Evolution of both male search (blue) and female
choice (red) under different combinations of the mortality rates
\(\mu_{\mathrm{in}}\) and \(\mu_{\mathrm{out}}\) (mortality in time-in
and out, respectively). The y-axis is the threshold fitness that leads
to evolution of male search (blue) or female choice (red). The results
show noise, but no correlation between the value of the mortality
parameters and the propensity for male search and/or female choice to
evolve. For each of the \(25 \times 20\) combinations of
\(\mu_{\mathrm{in}}\) and \(\mu_{\mathrm{out}}\), 1600 replicate
simulations were run.}
\end{figure}

\captionsetup{labelformat=default}

\clearpage

\captionsetup{labelformat=empty}

\begin{figure}
\centering
\includegraphics{ms_biorxiv_files/figure-latex/unnamed-chunk-7-1.pdf}
\caption{Figure S2.2 (\(T_{\mathrm{f}} = 0.5\)): The coevolution of male
search and female choosiness as a function of nuptial gift search time
(\(\alpha\)). Points show where the lower 95\% confidence interval of
female choosiness (red) and male search (blue) exceeds zero, indicating
evolution of choosiness or nuptial gift search. Each point includes data
from approximately 1600 replicate simulations with identical starting
conditions (some parameter combinations have fewer replicates). Red and
blue lines show thresholds above which the mathematical model predicts
that females should be choosy and males should search, respectively. Up
to 3000 interactions occur between individuals in each time step
(\(\psi = 3\)), potentially resulting in a mating interaction. The
number of individuals in the population remained at or near carrying
capacity of \(K = 1000\). Expected female processing time was set to
\(T_{\mathrm{f}}=0.5\) time steps, and \(\gamma\) and \(\alpha\) values
in the range {[}0.5, 1.5{]} and {[}0.1, 2.1{]}, respectively, were
used.}
\end{figure}

\captionsetup{labelformat=default}

\clearpage

\captionsetup{labelformat=empty}

\begin{figure}
\centering
\includegraphics{ms_biorxiv_files/figure-latex/unnamed-chunk-8-1.pdf}
\caption{Figure S2.3 (\(T_{\mathrm{f}} = 10.0\)): The coevolution of
male search and female choosiness as a function of nuptial gift search
time (\(\alpha\)). Points show where the lower 95\% confidence interval
of female choosiness (red) and male search (blue) exceeds zero,
indicating evolution of choosiness or nuptial gift search. Each point
includes data from approximately 1600 replicate simulations with
identical starting conditions (some parameter combinations include fewer
replicates). Red and blue lines show thresholds above which the
mathematical model predicts that females should be choosy and males
should search, respectively. Up to 3000 interactions occur between
individuals in each time step (\(\psi = 3\)), potentially resulting in a
mating interaction. The number of individuals in the population remained
at or near carrying capacity of \(K = 1000\). Expected female processing
time was set to \(T_{\mathrm{f}}=10.0\) time steps, and \(\gamma\) and
\(\alpha\) values in the range {[}0.5, 1.5{]} and {[}0.1, 2.1{]},
respectively, were used.}
\end{figure}

\captionsetup{labelformat=default}

\clearpage

\captionsetup{labelformat=empty}

\begin{figure}
\centering
\includegraphics{ms_biorxiv_files/figure-latex/unnamed-chunk-9-1.pdf}
\caption{Figure S2.4 (\(\psi = 1\)): The coevolution of male search and
female choosiness as a function of nuptial gift search time
(\(\alpha\)). Points show where the lower 95\% confidence interval of
where male search (blue) exceeds zero, indicating evolution of
choosiness or nuptial gift search. Each point includes data from
approximately 100 replicate simulations with identical starting
conditions (some parameter combinations had fewer replicates, minimum
400). Red and blue lines show thresholds above which the mathematical
model predicts that females should be choosy and males should search,
respectively. Up to 1000 interactions occur between individuals in each
time step (\(\psi = 1\)), potentially resulting in a mating interaction.
The number of individuals in the population remained at or near carrying
capacity of \(K = 1000\). Expected female processing time was set to
\(T_{\mathrm{f}}=2\) time steps, and \(\gamma\) and \(\alpha\) values in
the range {[}0.0, 1.4{]} and {[}0.1, 1.2{]}, respectively, were used.}
\end{figure}

\captionsetup{labelformat=default}

\clearpage

\captionsetup{labelformat=empty}

\begin{figure}
\centering
\includegraphics{ms_biorxiv_files/figure-latex/unnamed-chunk-10-1.pdf}
\caption{Figure S2.5 (\(\psi = 2\)): The coevolution of male search and
female choosiness as a function of nuptial gift search time
(\(\alpha\)). Points show where the lower 95\% confidence interval of
where male search (blue) exceeds zero, indicating evolution of
choosiness or nuptial gift search. Each point includes data from 100
replicate simulations with identical starting conditions (some parameter
combinations have fewer replicates). Red and blue lines show thresholds
above which the mathematical model predicts that females should be
choosy and males should search, respectively. Up to 2000 interactions
occur between individuals in each time step (\(\psi = 2\)), potentially
resulting in a mating interaction. The number of individuals in the
population remained at or near carrying capacity of \(K = 1000\).
Expected female processing time was set to \(T_{\mathrm{f}}=2\) time
steps, and \(\gamma\) and \(\alpha\) values in the range {[}0.0, 1.4{]}
and {[}0.1, 1.2{]}, respectively, were used.}
\end{figure}

\captionsetup{labelformat=default}

\clearpage

\captionsetup{labelformat=empty}

\begin{figure}
\centering
\includegraphics{ms_biorxiv_files/figure-latex/unnamed-chunk-11-1.pdf}
\caption{Figure S2.6 (\(\psi = 4\)): The coevolution of male search and
female choosiness as a function of nuptial gift search time
(\(\alpha\)). Points show where the lower 95\% confidence interval of
where male search (blue) exceeds zero, indicating evolution of
choosiness or nuptial gift search. Each point includes data from 100
replicate simulations with identical starting conditions (some parameter
combinations have fewer replicates). Red and blue lines show thresholds
above which the mathematical model predicts that females should be
choosy and males should search, respectively. Up to 4000 interactions
occur between individuals in each time step, potentially resulting in a
mating interaction (\(\psi = 4\)). The number of individuals in the
population remained at or near carrying capacity of \(K = 1000\).
Expected female processing time was set to \(T_{\mathrm{f}}=2\) time
steps, and \(\gamma\) and \(\alpha\) values in the range {[}0.0, 1.4{]}
and {[}0.1, 1.2{]}, respectively, were used.}
\end{figure}

\captionsetup{labelformat=default}

\clearpage

\captionsetup{labelformat=empty}

\begin{figure}
\centering
\includegraphics{ms_biorxiv_files/figure-latex/unnamed-chunk-12-1.pdf}
\caption{Figure S2.7 (\(\psi = 6\)): The coevolution of male search and
female choosiness as a function of nuptial gift search time
(\(\alpha\)). Points show where the lower 95\% confidence interval of
where male search (blue) exceeds zero, indicating evolution of
choosiness or nuptial gift search. Each point includes data from 100
replicate simulations with identical starting conditions (some parameter
combinations have fewer replicates). Red and blue lines show thresholds
above which the mathematical model predicts that females should be
choosy and males should search, respectively. Up to 6000 interactions
occur between individuals in each time step, potentially resulting in a
mating interaction (\(\psi = 6\)). The number of individuals in the
population remained at or near carrying capacity of \(K = 1000\).
Expected female processing time was set to \(T_{\mathrm{f}}=2\) time
steps, and \(\gamma\) and \(\alpha\) values in the range {[}0.0, 1.4{]}
and {[}0.1, 1.2{]}, respectively, were used.}
\end{figure}

\captionsetup{labelformat=default}

\clearpage

\captionsetup{labelformat=empty}

\begin{figure}
\centering
\includegraphics{ms_biorxiv_files/figure-latex/unnamed-chunk-13-1.pdf}
\caption{Figure S2.8 (\(T_{\mathrm{m}} = 0.5\)): The coevolution of male
search and female choosiness as a function of nuptial gift search time
(\(\alpha\)). Points show where the lower 95\% confidence interval of
where male search (blue) exceeds zero, indicating evolution of
choosiness or nuptial gift search. Each point includes data from 1600
replicate simulations with identical starting conditions. The red line
shows the threshold above which the mathematical model predicts that
females should be choosy (male thresholds for search are not shown
because simulations were initialised with males already searching for
nuptial gifts). Up to 3000 interactions occur between individuals in
each time step, potentially resulting in a mating interaction
(\(\psi = 3\)). The number of individuals in the population remained at
or near carrying capacity of \(K = 1000\). Expected female processing
time was set to \(T_{\mathrm{f}}=2\) time steps, and \(\gamma\) and
\(\alpha\) values in the range {[}0.0, 1.4{]} and {[}0.1, 1.2{]},
respectively, were used.}
\end{figure}

\captionsetup{labelformat=default}

\clearpage

\captionsetup{labelformat=empty}

\begin{figure}
\centering
\includegraphics{ms_biorxiv_files/figure-latex/unnamed-chunk-14-1.pdf}
\caption{Figure S2.9 (\(T_{\mathrm{m}} = 2.5\)): The coevolution of male
search and female choosiness as a function of nuptial gift search time
(\(\alpha\)). Points show where the lower 95\% confidence interval of
where male search (blue) exceeds zero, indicating evolution of
choosiness or nuptial gift search. Each point includes data from 1600
replicate simulations with identical starting conditions. The red line
shows the threshold above which the mathematical model predicts that
females should be choosy (male thresholds for search are not shown
because simulations were initialised with males already searching for
nuptial gifts). Up to 3000 interactions occur between individuals in
each time step, potentially resulting in a mating interaction
(\(\psi = 3\)). The number of individuals in the population remained at
or near carrying capacity of \(K = 1000\). Expected female processing
time was set to \(T_{\mathrm{f}}=2\) time steps, and \(\gamma\) and
\(\alpha\) values in the range {[}0.0, 1.4{]} and {[}0.1, 1.2{]},
respectively, were used.}
\end{figure}

\captionsetup{labelformat=default}

\clearpage

\hypertarget{s3-alternative-derivation-of-male-fitness-threshold}{%
\subsection{S3: Alternative derivation of male fitness
threshold}\label{s3-alternative-derivation-of-male-fitness-threshold}}

In the main text, we assumed that males made the decision to search or
not search for a nuptial gift. The expected length of time for which
searching males are expected to remain outside of the mating pool is
\(E[T_{\mathrm{m}}] = \alpha\) (see Methods). Alternatively, we can
assume that males search for a period of \(T_{\mathrm{m}}\) and spend
this full duration of \(T_{\mathrm{m}}\) in the time-out phase, even if
they succeed in finding a nuptial gift. The probability that a male
obtains a nuptial gift during this time is modelled in Eq. 1,

\[P(G) = 1 - e^{-\frac{1}{\alpha}T_{\mathrm{m}}}.\]

In Eq. 1, \(\alpha\) is the amount of time expected to pass before a
male encounters a nuptial gift. We assume that a male will only enter
the mating pool with no gift if they are unsuccessful in obtaining a
gift, so the probability that a male obtains no gift after
\(T_{\mathrm{m}}\) is,

\[P(L) = e^{-\frac{1}{\alpha}T_{\mathrm{m}}}.\]

We assume that the fitness increments to offspring associated with
receiving a nuptial gift versus no nuptial gift are \(1 + \gamma\) and
1, respectively. The rate at which males increase their fitness can then
be defined as the expected fitness increment from their nuptial gift
search divided by \(T_{\mathrm{m}}\) plus the time spent in the mating
pool waiting to encounter a mate,

\[W_{\mathrm{m}} = \lambda \frac{P(G)\left(1 + \gamma\right) + P(L)}{T_{\mathrm{m}} + \left( \frac{\beta + 1}{R} \right)}.\]

Our objective now is to determine the conditions under which a focal
male increases its fitness by searching for a nuptial gift
(\(T_{\mathrm{m}}>0\)) in a population of resident males that do not
search (\(T_{\mathrm{m}}=0\)). Females are assumed to exhibit no choice
in males with versus without nuptial gifts. Under such conditions, male
fitness cannot be affected by female choice, so selection to increase
\(T_{\mathrm{m}}>0\) must be based solely on \(\alpha\), \(\beta\),
\(R\), and \(\gamma\).

To determine under what conditions male inclusive fitness increases with
nuptial gift search time, we can differentiate \(W_{\mathrm{m}}\) with
respect to \(T_{\mathrm{m}}\),

\[\frac{\partial W_{\mathrm{m}}}{\partial T_{\mathrm{m}}} = \lambda\frac{\gamma\left(\frac{\frac{T_{\mathrm{m}} + \frac{\beta + 1}{R}}{\alpha} + 1}{e^{\frac{1}{\alpha}T_{\mathrm{m}}}} - 1\right) - 1}{\left(T_{\mathrm{m}} + \frac{\beta + 1}{R} \right)^{2}}.\]

Because \(T_{\mathrm{m}} = 0\), the above simplifies,

\[\frac{\partial W_{\mathrm{m}}}{\partial T_{\mathrm{m}}} = \lambda \frac{\frac{R\gamma\left(\beta + 1\right)}{\alpha} - R^{2}}{\left(1 + \beta \right)^{2}}.\]

We can re-arrange the above,

\[\frac{\partial W_{\mathrm{m}}}{\partial T_{\mathrm{m}}} = \lambda \frac{R\gamma}{\alpha\left(\beta+1\right)} - \lambda\frac{R^{2}}{{\left(1 + \beta \right)^{2}}}.\]

Note that if \(R = 0\) or \(\lambda = 0\), then, trivially, no change in
fitness occurs (since females and males cannot mate or do not produce
offspring). Fitness is increased by searching for nuptial gifts when
\(\gamma\) is high, scaled by the expect search time needed to find a
nuptial gift. A second term on the right-hand side is subtracted, which
reflects a loss in fitness proportional to the encounter rate of
potential mates in the mating pool. The conditions under which male
inclusive fitness increases by searching for a nuptial gift are found by
setting \(\partial W_{\mathrm{m}}/\partial T_{\mathrm{m}} = 0\) and
solving for \(\gamma\) to get Eq. 2 in the main text.

\clearpage

\hypertarget{s4-operational-sex-ratio}{%
\subsection{S4: Operational sex ratio}\label{s4-operational-sex-ratio}}

We assume that the ratio of males to females is the same upon individual
maturation. Consequently, the operational sex ratio \(\beta\) will be a
function of \(R\), \(T_{\mathrm{f}}\), and \(T_{\mathrm{m}}\) because
these parameters determine the density of females and males in the
mating pool versus outside of the mating pool. We start with the
definition of \(\beta\) as being the probability of finding an
individual in time-in (\protect\hyperlink{ref-Kokko2001}{Kokko \&
Monaghan, 2001}),

\[\beta = \frac{\int_{t=0}^{\infty}P_{IM}(t)dt}{\int_{t=0}^{\infty}P_{IF}(t)dt}\]

We can substitute the equations for \(P_{IM}(t)\) and \(P_{IF}(t)\),
which define the probabilities of males and females being within the
mating pool at time \(t\), respectively.

We can therefore calculate \(\beta\) as below,

\[\beta = \frac{\left( \frac{\left(\frac{\beta + 1}{R}\right)}{T_{\mathrm{m}} + \left(\frac{\beta + 1}{R}\right)} \right)}{\left( \frac{\left(R \frac{\beta}{\beta + 1}\right)}{T_{\mathrm{f}} + \left(R \frac{\beta}{\beta + 1}\right)} \right)}.\]

This can be simplified,

\[\beta = \frac{\left(\beta\left(R + T_{f}\right) + T_{f}\right)\left(\beta + 1\right)}{\beta \left(R^{2}T_{m} + R\right) + \beta^{2}R}.\]

There is no closed form solution for \(\beta\), but a recursive
algorithm can be used to calculate \(\beta\) to an arbitrary degree of
precision.

\begin{Shaded}
\begin{Highlighting}[]
\NormalTok{recursive\_b }\OtherTok{\textless{}{-}} \ControlFlowTok{function}\NormalTok{(B, D, Tf, Tm, }\AttributeTok{crit =} \FloatTok{0.0001}\NormalTok{, }\AttributeTok{maxit =} \DecValTok{9999}\NormalTok{)\{}
\NormalTok{  conv }\OtherTok{\textless{}{-}} \DecValTok{1}\NormalTok{;}
\NormalTok{  iter }\OtherTok{\textless{}{-}} \DecValTok{0}\NormalTok{;}
  \ControlFlowTok{while}\NormalTok{(conv }\SpecialCharTok{\textgreater{}}\NormalTok{ crit }\SpecialCharTok{\&}\NormalTok{ iter }\SpecialCharTok{\textless{}}\NormalTok{ maxit)\{}
\NormalTok{    Fe   }\OtherTok{\textless{}{-}}\NormalTok{ D }\SpecialCharTok{*}\NormalTok{ (B }\SpecialCharTok{/}\NormalTok{ (}\DecValTok{1} \SpecialCharTok{+}\NormalTok{ B));}
\NormalTok{    Me   }\OtherTok{\textless{}{-}}\NormalTok{ (}\DecValTok{1} \SpecialCharTok{+}\NormalTok{ B) }\SpecialCharTok{/}\NormalTok{ D;}
\NormalTok{    Bn   }\OtherTok{\textless{}{-}}\NormalTok{ (Me }\SpecialCharTok{/}\NormalTok{ (Tm }\SpecialCharTok{+}\NormalTok{ Me)) }\SpecialCharTok{/}\NormalTok{ (Fe }\SpecialCharTok{/}\NormalTok{ (Tf }\SpecialCharTok{+}\NormalTok{ Fe));}
\NormalTok{    iter }\OtherTok{\textless{}{-}}\NormalTok{ iter }\SpecialCharTok{+} \DecValTok{1}\NormalTok{;}
\NormalTok{    conv }\OtherTok{\textless{}{-}} \FunctionTok{abs}\NormalTok{(Bn }\SpecialCharTok{{-}}\NormalTok{ B);}
\NormalTok{    B    }\OtherTok{\textless{}{-}}\NormalTok{ Bn;}
\NormalTok{  \}}
  \FunctionTok{return}\NormalTok{(}\FunctionTok{list}\NormalTok{(}\AttributeTok{B =}\NormalTok{ B, }\AttributeTok{conv =}\NormalTok{ conv, }\AttributeTok{iter =}\NormalTok{ iter));}
\NormalTok{\}}
\end{Highlighting}
\end{Shaded}

We used the above function to calculate values of \(\beta\) for the
analytical model.

\clearpage

\hypertarget{s5-estimation-of-key-model-parameters-using-experimental-data}{%
\subsection{S5: Estimation of key model parameters using experimental
data}\label{s5-estimation-of-key-model-parameters-using-experimental-data}}

Estimates showing the mean number of offspring produced by female
\emph{Pisaura mirabilis} that ate nuptial gifts and females who did not.
Means were calculated with raw data from Tuni \emph{et al.}
(\protect\hyperlink{ref-Tuni2013a}{2013}) and results are shown \(\pm\)
SE (Table 1). Under these assumptions, the relative gain in fitness from
receiving nuptial gifts for a female is,

\[\hat{\delta_\mathrm{f}} = \frac{25.74}{6.00} = 4.29\]

Since the baseline fitness is 1, the increase in fitness resulting from
a nuptial gift then becomes,

\[\hat{\gamma} = \hat{\delta_{\mathrm{f}}} - 1 = 3.29.\]

The value 3.29 was used to parameterise \(\gamma\) for a set of
simulations (Figure 4 in the main text).

\begin{longtable}[]{@{}
  >{\raggedright\arraybackslash}p{(\columnwidth - 4\tabcolsep) * \real{0.4024}}
  >{\raggedright\arraybackslash}p{(\columnwidth - 4\tabcolsep) * \real{0.2805}}
  >{\raggedright\arraybackslash}p{(\columnwidth - 4\tabcolsep) * \real{0.3171}}@{}}
\toprule
\begin{minipage}[b]{\linewidth}\raggedright
\end{minipage} & \begin{minipage}[b]{\linewidth}\raggedright
Received nuptial gift
\end{minipage} & \begin{minipage}[b]{\linewidth}\raggedright
Received no nuptial gift
\end{minipage} \\
\midrule
\endhead
Expected number of hatched eggs & \(25.74 \pm 0.96\) &
\(6.00 \pm 2.1\) \\
\bottomrule
\end{longtable}

Table 1: Estimates showing mean number of offspring produced by female
\emph{P. mirabilis} that ate nuptial gifts and females who did not.
Means were calculated with raw data from Tuni \emph{et al.}
(\protect\hyperlink{ref-Tuni2013a}{2013}) and results are shown \(\pm\)
SE.

\clearpage

\hypertarget{s6-separate-evolution-of-male-search-and-female-choice}{%
\subsection{S6: Separate evolution of male search and female
choice}\label{s6-separate-evolution-of-male-search-and-female-choice}}

We used the individual-based simulation model (see Supporting
Information S1) to unpack the effect of coevolution on the evolution of
male search and female choice. Here we replicated the simulations shown
in the main text under the condition where only one trait at a time was
allowed to evolve and studied how this affected the trait evolution.

First, we submitted a set of simulations wherein male search did not
evolve, but was fixed at different values (see Fig. S6.1). Next, we ran
the same set of simulations wherein male search evolved, but female
choice was not possible. The results thus show how each trait evolves in
the absence of any coevolution (Fig. S6.1).

\captionsetup{labelformat=empty}

\begin{figure}
\centering
\includegraphics{ms_biorxiv_files/figure-latex/unnamed-chunk-16-1.pdf}
\caption{Figure S6.1: The separate evolution of male search and female
choosiness as a function of nuptial gift search time. Points show where
the lower 95\% confidence interval of male search (blue) and female
choosiness (red) exceeds zero, indicating evolution of nuptial gift
search or choosiness. Each point includes data from \(2 \times 1600\)
replicate simulations with identical starting conditions. In the first
batch, male search was constant and initialized at
\(T_{\mathrm{m}} = \alpha\), and female choice was evolving. In the
second batch, male search was evolving, and there was no option for
female choice. The parameters \(T_{\mathrm{f}}=2\), and \(\gamma\) and
\(\alpha\) values were set within the range {[}0.1, 2.0{]} and {[}0.3,
1.7{]}, respectively.}
\end{figure}

\captionsetup{labelformat=default}

\end{document}
